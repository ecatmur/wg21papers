%% Common header for WG21 proposals ? mainly taken from C++ standard draft source
%%

%%--------------------------------------------------
%% basics
\documentclass[a4paper,11pt,oneside,openany,final,article]{memoir}

\usepackage[american]
           {babel}        % needed for iso dates
\usepackage[iso,american]
           {isodate}      % use iso format for dates
\usepackage[final]
           {listings}     % code listings
\usepackage{longtable}    % auto-breaking tables
\usepackage{ltcaption}    % fix captions for long tables
\usepackage{relsize}      % provide relative font size changes
\usepackage{textcomp}     % provide \text{l,r}angle
\usepackage{underscore}   % remove special status of '_' in ordinary text
\usepackage{parskip}      % handle non-indented paragraphs "properly"
\usepackage{array}        % new column definitions for tables
\usepackage[normalem]{ulem}
\usepackage{enumitem}
\usepackage{color}        % define colors for strikeouts and underlines
\usepackage{xcolor}    % needed for blue links

\usepackage{amsmath}      % additional math symbols
\usepackage{mathrsfs}     % mathscr font
\usepackage[final]{microtype}
\usepackage{multicol}
\usepackage{lmodern}
\usepackage[T1]{fontenc}
\usepackage[pdftex, final]{graphicx}
\usepackage[pdftex,
            bookmarks=true,
            bookmarksnumbered=true,
            pdfpagelabels=true,
            pdfpagemode=UseOutlines,
            pdfstartview=FitH,
            linktocpage=true,
            colorlinks=true,
            plainpages=false,
            allcolors={blue}, 
            allbordercolors={white}
           ]{hyperref}
\usepackage{memhfixc}     % fix interactions between hyperref and memoir
\usepackage{url}  % urls in ref.bib
\usepackage{tabularx}  % don't use the C++ standard's fancy tables, they come with captions!

\pdfminorversion=5
\pdfcompresslevel=9
\pdfobjcompresslevel=2

\renewcommand\RSsmallest{5.5pt}  % smallest font size for relsize

%%--------------------------------------------------
%%--------------------------------------------------
%% Layout -- set overall page appearance

%%--------------------------------------------------
%%  set page size, type block size, type block position

\setlrmarginsandblock{2.245cm}{2.245cm}{*}
\setulmarginsandblock{2.5cm}{2.5cm}{*}

%%--------------------------------------------------
%%  set header and footer positions and sizes

\setheadfoot{\onelineskip}{2\onelineskip}
\setheaderspaces{*}{2\onelineskip}{*}

%%--------------------------------------------------
%%  make miscellaneous adjustments, then finish the layout
\setmarginnotes{7pt}{7pt}{0pt}
\checkandfixthelayout

%%--------------------------------------------------
%% If there is insufficient stretchable vertical space on a page,
%% TeX will not properly consider penalties for a good page break,
%% even if \raggedbottom (default) is in effect.
\addtolength{\topskip}{0pt plus 20pt}

%%--------------------------------------------------
%% Paragraph and bullet numbering

\newcounter{Paras}
\counterwithout{section}{chapter}
\setcounter{secnumdepth}{3}

\newcounter{Bullets1}[Paras]
\newcounter{Bullets2}[Bullets1]
\newcounter{Bullets3}[Bullets2]
\newcounter{Bullets4}[Bullets3]

\makeatletter
\newcommand{\parabullnum}[2]{%
\stepcounter{#1}%
\noindent\makebox[0pt][l]{\makebox[#2][r]{%
\scriptsize\raisebox{.7ex}%
{%
\ifnum \value{Paras}>0
\ifnum \value{Bullets1}>0 (\fi%
                          \arabic{Paras}%
\ifnum \value{Bullets1}>0 .\arabic{Bullets1}%
\ifnum \value{Bullets2}>0 .\arabic{Bullets2}%
\ifnum \value{Bullets3}>0 .\arabic{Bullets3}%
\fi\fi\fi%
\ifnum \value{Bullets1}>0 )\fi%
\fi%
}%
\hspace{\@totalleftmargin}\quad%
}}}
\makeatother

\def\pnum{\parabullnum{Paras}{0pt}}

%%--------------------------------------------------
%%--------------------------------------------------
%% Styles
%!TEX root = std.tex
%% styles.tex -- set styles for:
%     chapters
%     pages
%     footnotes

%%--------------------------------------------------
%%  create chapter style

\makechapterstyle{cppstd}{%
  \renewcommand{\beforechapskip}{\onelineskip}
  \renewcommand{\afterchapskip}{\onelineskip}
  \renewcommand{\chapternamenum}{}
  \renewcommand{\chapnamefont}{\chaptitlefont}
  \renewcommand{\chapnumfont}{\chaptitlefont}
  \renewcommand{\printchapternum}{\chapnumfont\thechapter\quad}
  \renewcommand{\afterchapternum}{}
}

%%--------------------------------------------------
%%  create page styles




%%--------------------------------------------------
% set style for main text
\setlength{\parindent}{0pt}
\setlength{\parskip}{1ex}

%%--------------------------------------------------
%% change list item markers to number and em-dash

\renewcommand{\labelitemi}{---\parabullnum{Bullets1}{\labelsep}}
\renewcommand{\labelitemii}{---\parabullnum{Bullets2}{\labelsep}}
\renewcommand{\labelitemiii}{---\parabullnum{Bullets3}{\labelsep}}
\renewcommand{\labelitemiv}{---\parabullnum{Bullets4}{\labelsep}}



%%--------------------------------------------------
%% override some functions from the listings package to avoid bad page breaks
%% (copied verbatim from listings.sty version 1.6 except where commented)
\makeatletter

\def\lst@Init#1{%
    \begingroup
    \ifx\lst@float\relax\else
        \edef\@tempa{\noexpand\lst@beginfloat{lstlisting}[\lst@float]}%
        \expandafter\@tempa
    \fi
    \ifx\lst@multicols\@empty\else
        \edef\lst@next{\noexpand\multicols{\lst@multicols}}
        \expandafter\lst@next
    \fi
    \ifhmode\ifinner \lst@boxtrue \fi\fi
    \lst@ifbox
        \lsthk@BoxUnsafe
        \hbox to\z@\bgroup
             $\if t\lst@boxpos \vtop
        \else \if b\lst@boxpos \vbox
        \else \vcenter \fi\fi
        \bgroup \par\noindent
    \else
        \lst@ifdisplaystyle
            \lst@EveryDisplay
            % make penalty configurable
            \par\lst@beginpenalty
            \vspace\lst@aboveskip
        \fi
    \fi
    \normalbaselines
    \abovecaptionskip\lst@abovecaption\relax
    \belowcaptionskip\lst@belowcaption\relax
    \lst@MakeCaption t%
    \lsthk@PreInit \lsthk@Init
    \lst@ifdisplaystyle
        \global\let\lst@ltxlabel\@empty
        \if@inlabel
            \lst@ifresetmargins
                \leavevmode
            \else
                \xdef\lst@ltxlabel{\the\everypar}%
                \lst@AddTo\lst@ltxlabel{%
                    \global\let\lst@ltxlabel\@empty
                    \everypar{\lsthk@EveryLine\lsthk@EveryPar}}%
            \fi
        \fi
        % A section heading might have set \everypar to apply a \clubpenalty
        % to the following paragraph, changing \everypar in the process.
        % Unconditionally overriding \everypar is a bad idea.
        % \everypar\expandafter{\lst@ltxlabel
        %                      \lsthk@EveryLine\lsthk@EveryPar}%
    \else
        \everypar{}\let\lst@NewLine\@empty
    \fi
    \lsthk@InitVars \lsthk@InitVarsBOL
    \lst@Let{13}\lst@MProcessListing
    \let\lst@Backslash#1%
    \lst@EnterMode{\lst@Pmode}{\lst@SelectCharTable}%
    \lst@InitFinalize}

\def\lst@DeInit{%
    \lst@XPrintToken \lst@EOLUpdate
    \global\advance\lst@newlines\m@ne
    \lst@ifshowlines
        \lst@DoNewLines
    \else
        \setbox\@tempboxa\vbox{\lst@DoNewLines}%
    \fi
    \lst@ifdisplaystyle \par\removelastskip \fi
    \lsthk@ExitVars\everypar{}\lsthk@DeInit\normalbaselines\normalcolor
    \lst@MakeCaption b%
    \lst@ifbox
        \egroup $\hss \egroup
        \vrule\@width\lst@maxwidth\@height\z@\@depth\z@
    \else
        \lst@ifdisplaystyle
            % make penalty configurable
            \par\lst@endpenalty
            \vspace\lst@belowskip
        \fi
    \fi
    \ifx\lst@multicols\@empty\else
        \def\lst@next{\global\let\@checkend\@gobble
                      \endmulticols
                      \global\let\@checkend\lst@@checkend}
        \expandafter\lst@next
    \fi
    \ifx\lst@float\relax\else
        \expandafter\lst@endfloat
    \fi
    \endgroup}


\def\lst@NewLine{%
    \ifx\lst@OutputBox\@gobble\else
        \par
        % add configurable penalties
        \lst@ifeolsemicolon
          \lst@semicolonpenalty
          \lst@eolsemicolonfalse
        \else
          \lst@domidpenalty
        \fi
        % Manually apply EveryLine and EveryPar; do not depend on \everypar
        \noindent \hbox{}\lsthk@EveryLine%
        % \lsthk@EveryPar uses \refstepcounter which balloons the PDF
    \fi
    \global\advance\lst@newlines\m@ne
    \lst@newlinetrue}

% new macro for empty lines, avoiding an \hbox that cannot be discarded
\def\lst@DoEmptyLine{%
  \ifvmode\else\par\fi\lst@emptylinepenalty
  \vskip\parskip
  \vskip\baselineskip
  % \lsthk@EveryLine has \lst@parshape, i.e. \parshape, which causes an \hbox
  % \lsthk@EveryPar increments line counters; \refstepcounter balloons the PDF
  \global\advance\lst@newlines\m@ne
  \lst@newlinetrue}

\def\lst@DoNewLines{
    \@whilenum\lst@newlines>\lst@maxempty \do
        {\lst@ifpreservenumber
            \lsthk@OnEmptyLine
            \global\advance\c@lstnumber\lst@advancelstnum
         \fi
         \global\advance\lst@newlines\m@ne}%
    \@whilenum \lst@newlines>\@ne \do
        % special-case empty printing of lines
        {\lsthk@OnEmptyLine\lst@DoEmptyLine}%
    \ifnum\lst@newlines>\z@ \lst@NewLine \fi}

% add keys for configuring before/end vertical penalties
\lst@Key{beginpenalty}\relax{\def\lst@beginpenalty{\penalty #1}}
\let\lst@beginpenalty\@empty
\lst@Key{midpenalty}\relax{\def\lst@midpenalty{\penalty #1}}
\let\lst@midpenalty\@empty
\lst@Key{endpenalty}\relax{\def\lst@endpenalty{\penalty #1}}
\let\lst@endpenalty\@empty
\lst@Key{emptylinepenalty}\relax{\def\lst@emptylinepenalty{\penalty #1}}
\let\lst@emptylinepenalty\@empty
\lst@Key{semicolonpenalty}\relax{\def\lst@semicolonpenalty{\penalty #1}}
\let\lst@semicolonpenalty\@empty

\lst@AddToHook{InitVars}{\let\lst@domidpenalty\@empty}
\lst@AddToHook{InitVarsEOL}{\let\lst@domidpenalty\lst@midpenalty}

% handle semicolons and closing braces (could be in \lstdefinelanguage as well)
\def\lst@eolsemicolontrue{\global\let\lst@ifeolsemicolon\iftrue}
\def\lst@eolsemicolonfalse{\global\let\lst@ifeolsemicolon\iffalse}
\lst@AddToHook{InitVars}{
  \global\let\lst@eolsemicolonpending\@empty
  \lst@eolsemicolonfalse
}
% If we found a semicolon or closing brace while parsing the current line,
% inform the subsequent \lst@NewLine about it for penalties.
\lst@AddToHook{InitVarsEOL}{%
  \ifx\lst@eolsemicolonpending\relax
    \lst@eolsemicolontrue
    \global\let\lst@eolsemicolonpending\@empty
  \fi%
}
\lst@AddToHook{SelectCharTable}{%
  % In theory, we should only detect trailing semicolons or braces,
  % but that would require un-doing the marking for any other character.
  % The next best thing is to undo the marking for closing parentheses,
  % because loops or if statements are the only places where we will
  % reasonably have a semicolon in the middle of a line, and those all
  % end with a closing parenthesis.
  \lst@DefSaveDef{41}\lstsaved@closeparen{%    handle closing parenthesis
    \lstsaved@closeparen
    \ifnum\lst@mode=\lst@Pmode    % regular processing mode (not a comment)
      \global\let\lst@eolsemicolonpending\@empty  % undo semicolon setting
    \fi%
  }%
  \lst@DefSaveDef{59}\lstsaved@semicolon{%     handle semicolon
    \lstsaved@semicolon
    \ifnum\lst@mode=\lst@Pmode    % regular processing mode (not a comment)
      \global\let\lst@eolsemicolonpending\relax
    \fi%
  }%
  \lst@DefSaveDef{125}\lstsaved@closebrace{%   handle closing brace
    \lst@eolsemicolonfalse        % do not break before a closing brace
    \lstsaved@closebrace          % might invoke \lst@NewLine
    \ifnum\lst@mode=\lst@Pmode    % regular processing mode (not a comment)
      \global\let\lst@eolsemicolonpending\relax
    \fi%
  }%
}

\makeatother


%%--------------------------------------------------
%%--------------------------------------------------
%% Macros
%!TEX root = std.tex
% Definitions and redefinitions of special commands

%%--------------------------------------------------
%% Difference markups
\definecolor{addclr}{rgb}{0,0.5,0.1}
\definecolor{remclr}{rgb}{1,0,0}
\definecolor{noteclr}{rgb}{0,0,1}

\renewcommand{\added}[1]{\textcolor{addclr}{\uline{#1}}}
\newcommand{\removed}[1]{\textcolor{remclr}{\sout{#1}}}
\renewcommand{\changed}[2]{\removed{#1}\added{#2}}

\newcommand{\nbc}[1]{[#1]\ }
\newcommand{\addednb}[2]{\added{\nbc{#1}#2}}
\newcommand{\removednb}[2]{\removed{\nbc{#1}#2}}
\newcommand{\changednb}[3]{\removednb{#1}{#2}\added{#3}}
\newcommand{\remitem}[1]{\item\removed{#1}}

\newcommand{\ednote}[1]{\textcolor{noteclr}{[Editor's note: #1] }}
% \newcommand{\ednote}[1]{}

\newenvironment{addedblock}
{
\color{addclr}
}
{
\color{black}
}
\newenvironment{removedblock}
{
\color{remclr}
}
{
\color{black}
}

%%--------------------------------------------------
% General code style
\newcommand{\CodeStyle}{\ttfamily}
\newcommand{\CodeStylex}[1]{\texttt{#1}}

% Code and definitions embedded in text.
\newcommand{\tcode}[1]{\CodeStylex{#1}}
\newcommand{\techterm}[1]{\textit{#1}}
\newcommand{\defnx}[2]{\indexdefn{#2}\textit{#1}}
\newcommand{\defn}[1]{\defnx{#1}{#1}}
\newcommand{\term}[1]{\textit{#1}}
\newcommand{\grammarterm}[1]{\textit{#1}}
\newcommand{\grammartermnc}[1]{\textit{#1}\nocorr}
\newcommand{\placeholder}[1]{\textit{#1}}
\newcommand{\placeholdernc}[1]{\textit{#1\nocorr}}

%%--------------------------------------------------
%% allow line break if needed for justification
\newcommand{\brk}{\discretionary{}{}{}}

%%--------------------------------------------------
%% Macros for funky text
\newcommand{\Cpp}{\texorpdfstring{C\kern-0.05em\protect\raisebox{.35ex}{\textsmaller[2]{+\kern-0.05em+}}}{C++}}
\newcommand{\CppIII}{\Cpp{} 2003}
\newcommand{\CppXI}{\Cpp{} 2011}
\newcommand{\CppXIV}{\Cpp{} 2014}
\newcommand{\CppXVII}{\Cpp{} 2017}
\newcommand{\opt}[1]{\ifthenelse{\equal{#1}{}}
    {\PackageError{main}{argument must not be empty}{}}
    {#1\ensuremath{_\mathit{opt}}}}
\newcommand{\dcr}{-{-}}
\newcommand{\bigoh}[1]{\ensuremath{\mathscr{O}(#1)}}

% Make all tildes a little larger to avoid visual similarity with hyphens.
\renewcommand{\~}{\textasciitilde}
\let\OldTextAsciiTilde\textasciitilde
\renewcommand{\textasciitilde}{\protect\raisebox{-0.17ex}{\larger\OldTextAsciiTilde}}
\newcommand{\caret}{\char`\^}

%%--------------------------------------------------
%% States and operators
\newcommand{\state}[2]{\tcode{#1}\ensuremath{_{#2}}}
\newcommand{\bitand}{\ensuremath{\, \mathsf{bitand} \,}}
\newcommand{\bitor}{\ensuremath{\, \mathsf{bitor} \,}}
\newcommand{\xor}{\ensuremath{\, \mathsf{xor} \,}}
\newcommand{\rightshift}{\ensuremath{\, \mathsf{rshift} \,}}
\newcommand{\leftshift}[1]{\ensuremath{\, \mathsf{lshift}_#1 \,}}

%% Notes and examples
\newcommand{\noteintro}[1]{[\,\textit{#1:}\space}
\newcommand{\noteoutro}[1]{\textit{\,---\,end #1}\,]}
\newenvironment{note}[1][Note]{\noteintro{#1}}{\noteoutro{note}\space}
\newenvironment{example}[1][Example]{\noteintro{#1}}{\noteoutro{example}\space}

%% Library function descriptions
\newcommand{\Fundescx}[1]{\textit{#1}}
\newcommand{\Fundesc}[1]{\Fundescx{#1:}\space}
\newcommand{\required}{\Fundesc{Required behavior}}
\newcommand{\requires}{\Fundesc{Requires}}
\newcommand{\effects}{\Fundesc{Effects}}
\newcommand{\postconditions}{\Fundesc{Postconditions}}
\newcommand{\returns}{\Fundesc{Returns}}
\newcommand{\throws}{\Fundesc{Throws}}
\newcommand{\default}{\Fundesc{Default behavior}}
\newcommand{\complexity}{\Fundesc{Complexity}}
\newcommand{\remarks}{\Fundesc{Remarks}}
\newcommand{\errors}{\Fundesc{Error conditions}}
\newcommand{\sync}{\Fundesc{Synchronization}}
\newcommand{\implimits}{\Fundesc{Implementation limits}}
\newcommand{\replaceable}{\Fundesc{Replaceable}}
\newcommand{\returntype}{\Fundesc{Return type}}
\newcommand{\cvalue}{\Fundesc{Value}}
\newcommand{\ctype}{\Fundesc{Type}}
\newcommand{\ctypes}{\Fundesc{Types}}
\newcommand{\dtype}{\Fundesc{Default type}}
\newcommand{\ctemplate}{\Fundesc{Class template}}
\newcommand{\templalias}{\Fundesc{Alias template}}

%% Cross reference
\newcommand{\xref}{\textsc{See also:}\space}

%% Inline parenthesized reference
\newcommand{\iref}[1]{\nolinebreak[3] (\ref{#1})}

%% NTBS, etc.
\newcommand{\NTS}[1]{\textsc{#1}}
\newcommand{\ntbs}{\NTS{ntbs}}
\newcommand{\ntmbs}{\NTS{ntmbs}}
% The following are currently unused:
% \newcommand{\ntwcs}{\NTS{ntwcs}}
% \newcommand{\ntcxvis}{\NTS{ntc16s}}
% \newcommand{\ntcxxxiis}{\NTS{ntc32s}}

%% Code annotations
\newcommand{\EXPO}[1]{\textit{#1}}
\newcommand{\expos}{\EXPO{exposition only}}
\newcommand{\impdef}{\EXPO{implementation-defined}}
\newcommand{\impdefnc}{\EXPO{implementation-defined\nocorr}}
\newcommand{\impdefx}[1]{\indeximpldef{#1}\EXPO{implementation-defined}}
\newcommand{\notdef}{\EXPO{not defined}}

\newcommand{\UNSP}[1]{\textit{\texttt{#1}}}
\newcommand{\UNSPnc}[1]{\textit{\texttt{#1}\nocorr}}
\newcommand{\unspec}{\UNSP{unspecified}}
\newcommand{\unspecnc}{\UNSPnc{unspecified}}
\newcommand{\unspecbool}{\UNSP{unspecified-bool-type}}
\newcommand{\seebelow}{\UNSP{see below}}
\newcommand{\seebelownc}{\UNSPnc{see below}}
\newcommand{\unspecuniqtype}{\UNSP{unspecified unique type}}
\newcommand{\unspecalloctype}{\UNSP{unspecified allocator type}}

\newcommand{\EXPLICIT}{\textit{\texttt{EXPLICIT}\nocorr}}

%% Manual insertion of italic corrections, for aligning in the presence
%% of the above annotations.
\newlength{\itcorrwidth}
\newlength{\itletterwidth}
\newcommand{\itcorr}[1][]{%
 \settowidth{\itcorrwidth}{\textit{x\/}}%
 \settowidth{\itletterwidth}{\textit{x\nocorr}}%
 \addtolength{\itcorrwidth}{-1\itletterwidth}%
 \makebox[#1\itcorrwidth]{}%
}

%% Double underscore
\newcommand{\ungap}{\kern.5pt}
\newcommand{\unun}{\textunderscore\ungap\textunderscore}
\newcommand{\xname}[1]{\tcode{\unun\ungap#1}}
\newcommand{\mname}[1]{\tcode{\unun\ungap#1\ungap\unun}}

%% An elided code fragment, /* ... */, that is formatted as code.
%% (By default, listings typeset comments as body text.)
%% Produces 9 output characters.
\newcommand{\commentellip}{\tcode{/* ...\ */}}

%% Ranges
\newcommand{\Range}[4]{\tcode{#1#3,\penalty2000{} #4#2}}
\newcommand{\crange}[2]{\Range{[}{]}{#1}{#2}}
\newcommand{\brange}[2]{\Range{(}{]}{#1}{#2}}
\newcommand{\orange}[2]{\Range{(}{)}{#1}{#2}}
\newcommand{\range}[2]{\Range{[}{)}{#1}{#2}}

%% Change descriptions
\newcommand{\diffdef}[1]{\hfill\break\textbf{#1:}\space}
\newcommand{\diffref}[1]{\pnum\textbf{Affected subclause:} \ref{#1}}
\newcommand{\change}{\diffdef{Change}}
\newcommand{\rationale}{\diffdef{Rationale}}
\newcommand{\effect}{\diffdef{Effect on original feature}}
\newcommand{\difficulty}{\diffdef{Difficulty of converting}}
\newcommand{\howwide}{\diffdef{How widely used}}

%% Miscellaneous
\newcommand{\uniquens}{\placeholdernc{unique}}
\newcommand{\stage}[1]{\item[Stage #1:]}
\newcommand{\doccite}[1]{\textit{#1}}
\newcommand{\cvqual}[1]{\textit{#1}}
\newcommand{\cv}{\cvqual{cv}}
\newcommand{\numconst}[1]{\textsl{#1}}
\newcommand{\logop}[1]{{\footnotesize #1}}

%%--------------------------------------------------
%% Environments for code listings.

% We use the 'listings' package, with some small customizations.  The
% most interesting customization: all TeX commands are available
% within comments.  Comments are set in italics, keywords and strings
% don't get special treatment.

\lstset{language=C++,
        basicstyle=\small\CodeStyle,
        keywordstyle=,
        stringstyle=,
        xleftmargin=1em,
        showstringspaces=false,
        commentstyle=\itshape\rmfamily,
        columns=fullflexible,
        keepspaces=true,
        texcl=true}

% Our usual abbreviation for 'listings'.  Comments are in
% italics.  Arbitrary TeX commands can be used if they're
% surrounded by @ signs.
\newcommand{\CodeBlockSetup}{
 \lstset{escapechar=@, aboveskip=\parskip, belowskip=0pt,
         midpenalty=500, endpenalty=-50,
         emptylinepenalty=-250, semicolonpenalty=0}
 \renewcommand{\tcode}[1]{\textup{\CodeStylex{##1}}}
 \renewcommand{\techterm}[1]{\textit{\CodeStylex{##1}}}
 \renewcommand{\term}[1]{\textit{##1}}
 \renewcommand{\grammarterm}[1]{\textit{##1}}
}

\lstnewenvironment{codeblock}{\CodeBlockSetup}{}

% An environment for command / program output that is not C++ code.
\lstnewenvironment{outputblock}{\lstset{language=}}{}

% A code block in which single-quotes are digit separators
% rather than character literals.
\lstnewenvironment{codeblockdigitsep}{
 \CodeBlockSetup
 \lstset{deletestring=[b]{'}}
}{}

% Permit use of '@' inside codeblock blocks (don't ask)
\makeatletter
\newcommand{\atsign}{@}
\makeatother

%%--------------------------------------------------
%% Indented text
\newenvironment{indented}[1][]
{\begin{indenthelper}[#1]\item\relax}
{\end{indenthelper}}

%%--------------------------------------------------
%% Library item descriptions
\lstnewenvironment{itemdecl}
{
 \lstset{escapechar=@,
 xleftmargin=0em,
 midpenalty=500,
 semicolonpenalty=-50,
 endpenalty=3000,
 aboveskip=2ex,
 belowskip=0ex	% leave this alone: it keeps these things out of the
				% footnote area
 }
}
{
}

\newenvironment{itemdescr}
{
 \begin{indented}[beginpenalty=3000, endpenalty=-300]}
{
 \end{indented}
}

%%--------------------------------------------------
%% add special hyphenation rules
\hyphenation{tem-plate ex-am-ple in-put-it-er-a-tor name-space name-spaces non-zero}

%%--------------------------------------------------
%% turn off all ligatures inside \texttt
\DisableLigatures{encoding = T1, family = tt*}






\begin{document}
\title{Do not remove the Lakos Rule before standardising Contracts}
\author{ Timur Doumler \small(\href{mailto:papers@timur.audio}{papers@timur.audio})  }
\date{}
\maketitle

\begin{tabular}{ll}
Document \#: & D2831R0 \\
Date: &2023-03-08 \\
Project: & Programming Language C++ \\
Audience: & Library Evolution Working Group
\end{tabular}

\begin{abstract}
The Lakos Rule is a longstanding design principle in the C++ standard library. It stipulates that a function with a narrow contract should not be  \tcode{noexcept}, even if it is known to not throw when called with valid input. In this paper, we present a case study showing why the Lakos Rule is still useful and important today, and why we should not remove it, at least not until we have standardised a C++ Contracts facility that offers a superior alternative.
\end{abstract}

\section{Introduction}
\label{sec:intro}

C++ functions -- in the standard library or in other places -- can have \emph{preconditions}, which are a form of \emph{contract}. A function that has no preconditions on its input values -- i.e. a function that has defined behaviour for any input values, is said to have a \emph{wide contract}.  If such a function is known to never throw an exception, it should be marked as \tcode{noexcept}. An example of such a function in the C++ standard library is \tcode{std::get(std::array)}.

By contrast, a function that has preconditions on its input values -- i.e. a function that has undefined behaviour for some input values, which are considered \emph{invalid} -- is said to have a \emph{narrow contract}. An example of a function with a narrow contract in the C++ standard library is \tcode{std::vector::operator[]}.

It has been a longstanding design principle in the C++ standard library that a function with a narrow contract should not be marked as \tcode{noexcept}, even if it is known to never throw an exception for \emph{valid} input values. In the latter case, it should merely be specified as ``Throws: nothing''. Not marking such a function as \tcode{noexcept} allows for testing strategies that involve throwing exceptions as a way of diagnosing \emph{contract violations} -- i.e. bugs introduced by calling the function with invalid input values (calling it \emph{out of contract}). This design principle is also known as the \emph{Lakos Rule}.

The Lakos Rule was first proposed in \cite{N3248}, then adopted with \cite{N3279}. An updated version of the rule was codified into policy in \cite{P0884R0}. See \cite{O'Dwyer2018} for a more detailed summary.

More recently, it has been argued in \cite{P1656R2} that the Lakos Rule should be abandoned as a design principle, and that functions which are known to never throw for valid input values should always be marked as \tcode{noexcept}, regardless of whether they have a wide or a narrow contract. It has been further proposed in \cite{P2148R0} to adopt a new standing document with design guidelines for the evolution of the C++
Standard Library that move away from the Lakos Rule.

In section \ref{sec:casestudy} of this paper, we present a case study to demonstrate that the Lakos Rule is still useful and important today, as it can help detect and prevent bugs in different kinds of code bases. In section \ref{sec:contracts} we argue further that a C++ Contracts facility might offer a superior solution to the problem of testing for contract violations, and that we might want to reconsider the Lakos Rule after we have standardised such a facility; but that we should not remove the Lakos Rule before standardising Contracts as there is currently no viable alternative for cross-platform codebases that rely on the Lakos Rule to test for contract violations.

\section{Case study}
\label{sec:casestudy}

\subsection{The premise}

In 2018, I co-founded the music technology company Cradle (\hyperref[https://cradle.app]{\tcode{https://cradle.app}}) and became its CTO. I was in the enviable position of being able to start a brand new code base from scratch, following the latest modern C++ best practices. I also hired a brand new developer team, with the freedom to introduce whatever software technology and coding guidelines made most sense for the thing we set out to accomplish.

From the start, the core guiding principle for building Cradle's software stack and engineering culture was a strong focus on code quality. One of the things we introduced to achieve this goal is to aim for a high unit test coverage. Focusing on automated testing in general, and unit testing in particular, tends to be less common in  music production software than in other industries. We found out in practice that, by having a strong culture of unit testing and test-driven development (TDD), we were able to deliver software at  a significantly higher quality standard, with much fewer bugs and crashes reported by users. The parts of our code base where TDD proved to be particularly effective were the foundational, generic C++ libraries that the rest of the codebase relied upon. In general, whenever feasible, we aim for every line of product code to have some corresponding test code.

\subsection{The problem}

However, as we started practicing TDD with modern C++, we immediately ran into the following problem. Let us consider the following C++ function as an example (a very typical function for a foundational audio software library):
\begin{codeblock}
float fast_log(float x);  // efficiently computes the base e logarithm of x
\end{codeblock}

This function is specified to have a narrow contract: it has a precondition that \tcode{x >= 0}. If this precondition is violated, the behaviour is undefined; this specification is necessary to achieve maximum performance in a release build. Calling \tcode{fast_log} with an argument less than \tcode{0} is therefore unconditionally a bug.

Now, writing unit tests to ensure that this function returns the correct result for valid input is straightforward. One such test might look like this:
\begin{codeblock}
CHECK_APPROX_EQUAL(fast_log(1), 0);
\end{codeblock}
where \tcode{CHECK_APPROX_EQUAL} is some macro provided by the unit test framework, checking that two floating-point numbers are approximately equal within some specified tolerance. But this is not the only kind of unit test that we need to write. We also need to ensure that the function behaves as intended when called out of contract.

In our case, the intended behaviour for invalid input is as follows: in debug mode, the precondition \tcode{x >= 0} should be checked, and a contract violation should trigger an assertion failure; in release mode, the precondition should be ignored, and a contract violation would in general lead to undefined behaviour. Checking the precondition in debug mode is critically important, as otherwise the developer might not realise they have called \tcode{fast_log} out of contract, and introduce a new bug to the code base.

Adding a precondition check in C++23, which lacks a Contracts facility, might look like this:
\begin{codeblock}
float fast_log(float x) {
  ASSERT(x >= 0);
  // implementation
}
\end{codeblock}
where \tcode{ASSERT} is a macro that prints a diagnostic and aborts the program in a debug build, and does nothing in a release build\footnote{In our codebase, there is a third case: whenever a debugger is attached, the macro calls a function that causes the debugger to break, very similar to \tcode{std::breakpoint()} as proposed in \cite{P2514R0}. This is one of the reasons why we, like many other codebases, have a custom \tcode{ASSERT} macro and do not simply use the \tcode{assert} macro from header \tcode{cassert}.}.

Following our principle that, whenever feasible, every line of product code should have some corresponding test code, when adding such a precondition check to \tcode{fast_log}, we also need to write a unit test to ensure that the precondition check has in fact been added:
\begin{codeblock}
CHECK_ASSERT_FAIL(fast_log(-1));
\end{codeblock}
But how do we write such a test?

\subsection{Solution: the Lakos Rule}
Once we hit the \tcode{ASSERT} macro, and the assertion fails, it is no longer meaningful to continue executing the body of the function; this will either crash or lead to other forms of undefined behaviour. Therefore, in order to continue running our unit test suite, we need a way to exit the function other than by returning a value, at the point where the \tcode{ASSERT} fails. 

The most natural way to do so is to throw an exception. This is exactly why the Lakos Rule exists: since \tcode{fast_log} has a narrow contract, we do not mark it as \tcode{noexcept}, and we can define our \tcode{ASSERT} macro as follows:

\begin{codeblock}
#if NDEBUG
  #define ASSERT(expr)
#else
  #define ASSERT(expr) if (!expr) throw AssertFail();
#endif
\end{codeblock}
Then, we can define our \tcode{CHECK_ASSERT_FAIL} as checking that an exception of type \tcode{AssertFail} has been thrown; every modern C++ unit testing framework provides this functionality.

\subsection{Alternatives}

If we abandon the Lakos Rule as a design principle, as proposed in \cite{P1656R2} and \cite{P2148R0}, writing such tests (and therefore preventing bugs due to missing asserts) becomes much harder. Without the Lakos Rule, \tcode{fast_log} ought to be marked as \tcode{noexcept}, since it is known to never throw when called with valid input. This makes writing the \tcode{ASSERT} and \tcode{CHECK_ASSERT_FAIL} macros as above impossible: throwing \tcode{AssertFail()} out of a function marked \tcode{noexcept} would call \tcode{std::terminate}, immediately bringing down the whole test suite. At Cradle, we experimented with the following workarounds, none of which were deemed satisfactory.

\subsubsection{Death tests}

In order to prevent \tcode{std::terminate} from bringing down the whole test suite, we can write our \tcode{CHECK_ASSERT_FAIL} as a death test. In a death test, the current process is forked, the code under test is run in the child process, and then the unit test framework checks that that child process has terminated abnormally, in which case the death test passes. This approach works in theory, but has several drawbacks:

\begin{itemize}
\item Most C++ unit test frameworks do not offer support for death tests. From the popular C++ unit test frameworks, only GoogleTest does, while Catch2, Boost.Test, CppTest, and DocTest do not. At Cradle, for various technical reasons we ended up using Catch2, which means that death tests were not available to us.
\item We found death tests to have a significant runtime overhead when compared to unit tests that merely check whether an exception has been thrown. Running all unit tests takes orders of magnitude more time when using death tests, significantly slowing down the CI pipieline.
\item Death tests lose information: the unit test framework can verify that the child process has terminated abnormally, but not whether this was actually caused by a contract violation (which is the thing we want to test).
\end{itemize}

\subsubsection{POSIX signals}

If we cannot use child processes and death tests, and we cannot use exceptions, another way to abort the execution of the function from our \tcode{ASSERT} macro and to signal a contract violation to our unit test framework is to raise a POSIX signal. This works well on POSIX platforms; however, at Cradle, we are deploying cross-platform audio software which should run (and therefore be tested on) all of macOS, Linux, and Windows. POSIX signals are not available on Windows, therefore this approach is not viable.

\subsubsection{\tcode{setjmp} and \tcode{longjmp}}

Yet another way to abort the execution of the function from our \tcode{ASSERT} macro is to use \tcode{setjmp} and \tcode{longjmp}. This has the disadvantage that the stack is not unwound. If we run thousands of unit tests invoving data structures that allocate significant amounts of memory on the heap, we end up with an unacceptable amount of memory leaks.

\subsubsection{Making \tcode{noexcept} conditional on whether we are in unit test mode}

Another workaround that we used at Cradle was to introduce a macro as follows:
\begin{codeblock}
#if UNIT_TEST_MODE
  #define NOEXCEPT 
#else
  #define NOEXCEPT noexcept
#endif
\end{codeblock}
Then, we can mark all functions with a narrow contract with this \tcode{NOEXCEPT} macro rather than the \tcode{noexcept} keyword, and compile the library with \tcode{-DUNIT_TEST_MODE=1} for the purposes of running unit tests. However, this is not satisfactory either, because in this way we effectively end up not unit testing our actual code, but unit testing code that is compiled differently and results in a different binary with different behaviour.

\section{Contracts}
\label{sec:contracts}

SG21 is currently working on standardising a \emph{Contracts facility}, i.e. a new language feature to be added to standard C++ that allows the user to express preconditions, postconditions, and assertions in C++ code. Attempts to standardise a Contracts facility have a long history; according to the current SG21 roadmap \cite{P2695R1} we are aiming to get a Contracts MVP into C++26. See \cite{P2521R3} and references therein for a summary of the current state of this effort.

With the Contracts MVP, we will be able to express the precondition of \tcode{fast_log} as follows (using one of the three options for syntax currently under consideration):
\begin{codeblock}
float fast_log(float x) [[ pre: x >= 0 ]];
\end{codeblock}

The current Contracts MVP proposes two build modes: \emph{No_eval}, in which the precondition is ignored, and \emph{Eval_and_abort}, in which the precondition is checked; if the predicate evaluates to \tcode{false}, \tcode{std::terminate} is called. At first, it seems that for purposes of testing for contract violations we have not gained much: calling \tcode{fast_log} out of contract will result in \tcode{std::terminate} being called, which leaves death tests as the only viable method to write tests for such a call.

However, the Contracts MVP is merely a first step towards having a standardised Contracts facility. The Contracts MVP is explicitly designed to be extended, such as by adding custom violation handlers (\cite{P2698R0}, \cite{P2811R0}) or other facilities (\cite{P2784R0}) which allow us to test out-of-contract calls without relying on the function under test not being marked as \tcode{noexcept}. Even if we only get the Contracts MVP in C++26, without custom violation handlers or similar facilities, the Contracts MVP is explicitly designed to allow compiler vendors to offer additional build modes where such behaviour can be configured. In either case, we will have a superior way to test for contract violations, and we will no longer need the Lakos Rule to enable this. However, as long as such a Contracts facility is not part of standard C++, there is no adequate replacement for the Lakos Rule, and we should therefore not remove it at this time.

\section{Conclusion}
At Cradle, we found that testing our code for contract violations has proven to be an important part of keeping our code quality high and reducing the amount of introduced bugs. We found that out of all methods to achieve this kind of testing, following the Lakos Rule is the most straightforward and effective method with the most favourable tradeoffs. 

We experimented with death tests, POSIX signals, \tcode{setjmp} and \tcode{longjmp}, and making \tcode{noexcept} conditional on whether we are in unit test mode. We found that all of these alternative approaches have worse tradeoffs than following the Lakos Rule for our use case.

Since the Lakos Rule works so well for foundational C++ libraries in the cross-platform software space, it is reasonable to assume that it would also be a viable approach for at least some implementations of the C++ standard library. We therefore urge the C++ committee to not abandon the Lakos Rule as a design principle for the standard library, at least not until we standardise a Contracts facility in C++.

A standard C++ Contracts facility might provide a superior way to test a codebase for contract violations, at which point the Lakos Rule might no longer be needed. However, changing our library design guidelines and removing the Lakos Rule should only be considered at that point, and not while no such superior facility exists in standard C++.

%\section*{Document history}

%\begin{itemize}
%\item \textbf{R0}, 2023-03-08: Initial version.
%\item \textbf{R1}, 20XX-XX-XX: ??
%\end{itemize}

\renewcommand{\bibname}{References}
\bibliographystyle{abstract}
\bibliography{ref}

\end{document}