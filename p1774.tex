%% Common header for WG21 proposals ? mainly taken from C++ standard draft source
%%

%%--------------------------------------------------
%% basics
\documentclass[a4paper,11pt,oneside,openany,final,article]{memoir}

\usepackage[american]
           {babel}        % needed for iso dates
\usepackage[iso,american]
           {isodate}      % use iso format for dates
\usepackage[final]
           {listings}     % code listings
\usepackage{longtable}    % auto-breaking tables
\usepackage{ltcaption}    % fix captions for long tables
\usepackage{relsize}      % provide relative font size changes
\usepackage{textcomp}     % provide \text{l,r}angle
\usepackage{underscore}   % remove special status of '_' in ordinary text
\usepackage{parskip}      % handle non-indented paragraphs "properly"
\usepackage{array}        % new column definitions for tables
\usepackage[normalem]{ulem}
\usepackage{enumitem}
\usepackage{color}        % define colors for strikeouts and underlines
\usepackage{xcolor}    % needed for blue links

\usepackage{amsmath}      % additional math symbols
\usepackage{mathrsfs}     % mathscr font
\usepackage[final]{microtype}
\usepackage{multicol}
\usepackage{lmodern}
\usepackage[T1]{fontenc}
\usepackage[pdftex, final]{graphicx}
\usepackage[pdftex,
            bookmarks=true,
            bookmarksnumbered=true,
            pdfpagelabels=true,
            pdfpagemode=UseOutlines,
            pdfstartview=FitH,
            linktocpage=true,
            colorlinks=true,
            plainpages=false,
            allcolors={blue}, 
            allbordercolors={white}
           ]{hyperref}
\usepackage{memhfixc}     % fix interactions between hyperref and memoir
\usepackage{url}  % urls in ref.bib
\usepackage{tabularx}  % don't use the C++ standard's fancy tables, they come with captions!

\pdfminorversion=5
\pdfcompresslevel=9
\pdfobjcompresslevel=2

\renewcommand\RSsmallest{5.5pt}  % smallest font size for relsize

%%--------------------------------------------------
%%--------------------------------------------------
%% Layout -- set overall page appearance

%%--------------------------------------------------
%%  set page size, type block size, type block position

\setlrmarginsandblock{2.245cm}{2.245cm}{*}
\setulmarginsandblock{2.5cm}{2.5cm}{*}

%%--------------------------------------------------
%%  set header and footer positions and sizes

\setheadfoot{\onelineskip}{2\onelineskip}
\setheaderspaces{*}{2\onelineskip}{*}

%%--------------------------------------------------
%%  make miscellaneous adjustments, then finish the layout
\setmarginnotes{7pt}{7pt}{0pt}
\checkandfixthelayout

%%--------------------------------------------------
%% If there is insufficient stretchable vertical space on a page,
%% TeX will not properly consider penalties for a good page break,
%% even if \raggedbottom (default) is in effect.
\addtolength{\topskip}{0pt plus 20pt}

%%--------------------------------------------------
%% Paragraph and bullet numbering

\newcounter{Paras}
\counterwithout{section}{chapter}
\setcounter{secnumdepth}{3}

\newcounter{Bullets1}[Paras]
\newcounter{Bullets2}[Bullets1]
\newcounter{Bullets3}[Bullets2]
\newcounter{Bullets4}[Bullets3]

\makeatletter
\newcommand{\parabullnum}[2]{%
\stepcounter{#1}%
\noindent\makebox[0pt][l]{\makebox[#2][r]{%
\scriptsize\raisebox{.7ex}%
{%
\ifnum \value{Paras}>0
\ifnum \value{Bullets1}>0 (\fi%
                          \arabic{Paras}%
\ifnum \value{Bullets1}>0 .\arabic{Bullets1}%
\ifnum \value{Bullets2}>0 .\arabic{Bullets2}%
\ifnum \value{Bullets3}>0 .\arabic{Bullets3}%
\fi\fi\fi%
\ifnum \value{Bullets1}>0 )\fi%
\fi%
}%
\hspace{\@totalleftmargin}\quad%
}}}
\makeatother

\def\pnum{\parabullnum{Paras}{0pt}}

%%--------------------------------------------------
%%--------------------------------------------------
%% Styles
%!TEX root = std.tex
%% styles.tex -- set styles for:
%     chapters
%     pages
%     footnotes

%%--------------------------------------------------
%%  create chapter style

\makechapterstyle{cppstd}{%
  \renewcommand{\beforechapskip}{\onelineskip}
  \renewcommand{\afterchapskip}{\onelineskip}
  \renewcommand{\chapternamenum}{}
  \renewcommand{\chapnamefont}{\chaptitlefont}
  \renewcommand{\chapnumfont}{\chaptitlefont}
  \renewcommand{\printchapternum}{\chapnumfont\thechapter\quad}
  \renewcommand{\afterchapternum}{}
}

%%--------------------------------------------------
%%  create page styles




%%--------------------------------------------------
% set style for main text
\setlength{\parindent}{0pt}
\setlength{\parskip}{1ex}

%%--------------------------------------------------
%% change list item markers to number and em-dash

\renewcommand{\labelitemi}{---\parabullnum{Bullets1}{\labelsep}}
\renewcommand{\labelitemii}{---\parabullnum{Bullets2}{\labelsep}}
\renewcommand{\labelitemiii}{---\parabullnum{Bullets3}{\labelsep}}
\renewcommand{\labelitemiv}{---\parabullnum{Bullets4}{\labelsep}}



%%--------------------------------------------------
%% override some functions from the listings package to avoid bad page breaks
%% (copied verbatim from listings.sty version 1.6 except where commented)
\makeatletter

\def\lst@Init#1{%
    \begingroup
    \ifx\lst@float\relax\else
        \edef\@tempa{\noexpand\lst@beginfloat{lstlisting}[\lst@float]}%
        \expandafter\@tempa
    \fi
    \ifx\lst@multicols\@empty\else
        \edef\lst@next{\noexpand\multicols{\lst@multicols}}
        \expandafter\lst@next
    \fi
    \ifhmode\ifinner \lst@boxtrue \fi\fi
    \lst@ifbox
        \lsthk@BoxUnsafe
        \hbox to\z@\bgroup
             $\if t\lst@boxpos \vtop
        \else \if b\lst@boxpos \vbox
        \else \vcenter \fi\fi
        \bgroup \par\noindent
    \else
        \lst@ifdisplaystyle
            \lst@EveryDisplay
            % make penalty configurable
            \par\lst@beginpenalty
            \vspace\lst@aboveskip
        \fi
    \fi
    \normalbaselines
    \abovecaptionskip\lst@abovecaption\relax
    \belowcaptionskip\lst@belowcaption\relax
    \lst@MakeCaption t%
    \lsthk@PreInit \lsthk@Init
    \lst@ifdisplaystyle
        \global\let\lst@ltxlabel\@empty
        \if@inlabel
            \lst@ifresetmargins
                \leavevmode
            \else
                \xdef\lst@ltxlabel{\the\everypar}%
                \lst@AddTo\lst@ltxlabel{%
                    \global\let\lst@ltxlabel\@empty
                    \everypar{\lsthk@EveryLine\lsthk@EveryPar}}%
            \fi
        \fi
        % A section heading might have set \everypar to apply a \clubpenalty
        % to the following paragraph, changing \everypar in the process.
        % Unconditionally overriding \everypar is a bad idea.
        % \everypar\expandafter{\lst@ltxlabel
        %                      \lsthk@EveryLine\lsthk@EveryPar}%
    \else
        \everypar{}\let\lst@NewLine\@empty
    \fi
    \lsthk@InitVars \lsthk@InitVarsBOL
    \lst@Let{13}\lst@MProcessListing
    \let\lst@Backslash#1%
    \lst@EnterMode{\lst@Pmode}{\lst@SelectCharTable}%
    \lst@InitFinalize}

\def\lst@DeInit{%
    \lst@XPrintToken \lst@EOLUpdate
    \global\advance\lst@newlines\m@ne
    \lst@ifshowlines
        \lst@DoNewLines
    \else
        \setbox\@tempboxa\vbox{\lst@DoNewLines}%
    \fi
    \lst@ifdisplaystyle \par\removelastskip \fi
    \lsthk@ExitVars\everypar{}\lsthk@DeInit\normalbaselines\normalcolor
    \lst@MakeCaption b%
    \lst@ifbox
        \egroup $\hss \egroup
        \vrule\@width\lst@maxwidth\@height\z@\@depth\z@
    \else
        \lst@ifdisplaystyle
            % make penalty configurable
            \par\lst@endpenalty
            \vspace\lst@belowskip
        \fi
    \fi
    \ifx\lst@multicols\@empty\else
        \def\lst@next{\global\let\@checkend\@gobble
                      \endmulticols
                      \global\let\@checkend\lst@@checkend}
        \expandafter\lst@next
    \fi
    \ifx\lst@float\relax\else
        \expandafter\lst@endfloat
    \fi
    \endgroup}


\def\lst@NewLine{%
    \ifx\lst@OutputBox\@gobble\else
        \par
        % add configurable penalties
        \lst@ifeolsemicolon
          \lst@semicolonpenalty
          \lst@eolsemicolonfalse
        \else
          \lst@domidpenalty
        \fi
        % Manually apply EveryLine and EveryPar; do not depend on \everypar
        \noindent \hbox{}\lsthk@EveryLine%
        % \lsthk@EveryPar uses \refstepcounter which balloons the PDF
    \fi
    \global\advance\lst@newlines\m@ne
    \lst@newlinetrue}

% new macro for empty lines, avoiding an \hbox that cannot be discarded
\def\lst@DoEmptyLine{%
  \ifvmode\else\par\fi\lst@emptylinepenalty
  \vskip\parskip
  \vskip\baselineskip
  % \lsthk@EveryLine has \lst@parshape, i.e. \parshape, which causes an \hbox
  % \lsthk@EveryPar increments line counters; \refstepcounter balloons the PDF
  \global\advance\lst@newlines\m@ne
  \lst@newlinetrue}

\def\lst@DoNewLines{
    \@whilenum\lst@newlines>\lst@maxempty \do
        {\lst@ifpreservenumber
            \lsthk@OnEmptyLine
            \global\advance\c@lstnumber\lst@advancelstnum
         \fi
         \global\advance\lst@newlines\m@ne}%
    \@whilenum \lst@newlines>\@ne \do
        % special-case empty printing of lines
        {\lsthk@OnEmptyLine\lst@DoEmptyLine}%
    \ifnum\lst@newlines>\z@ \lst@NewLine \fi}

% add keys for configuring before/end vertical penalties
\lst@Key{beginpenalty}\relax{\def\lst@beginpenalty{\penalty #1}}
\let\lst@beginpenalty\@empty
\lst@Key{midpenalty}\relax{\def\lst@midpenalty{\penalty #1}}
\let\lst@midpenalty\@empty
\lst@Key{endpenalty}\relax{\def\lst@endpenalty{\penalty #1}}
\let\lst@endpenalty\@empty
\lst@Key{emptylinepenalty}\relax{\def\lst@emptylinepenalty{\penalty #1}}
\let\lst@emptylinepenalty\@empty
\lst@Key{semicolonpenalty}\relax{\def\lst@semicolonpenalty{\penalty #1}}
\let\lst@semicolonpenalty\@empty

\lst@AddToHook{InitVars}{\let\lst@domidpenalty\@empty}
\lst@AddToHook{InitVarsEOL}{\let\lst@domidpenalty\lst@midpenalty}

% handle semicolons and closing braces (could be in \lstdefinelanguage as well)
\def\lst@eolsemicolontrue{\global\let\lst@ifeolsemicolon\iftrue}
\def\lst@eolsemicolonfalse{\global\let\lst@ifeolsemicolon\iffalse}
\lst@AddToHook{InitVars}{
  \global\let\lst@eolsemicolonpending\@empty
  \lst@eolsemicolonfalse
}
% If we found a semicolon or closing brace while parsing the current line,
% inform the subsequent \lst@NewLine about it for penalties.
\lst@AddToHook{InitVarsEOL}{%
  \ifx\lst@eolsemicolonpending\relax
    \lst@eolsemicolontrue
    \global\let\lst@eolsemicolonpending\@empty
  \fi%
}
\lst@AddToHook{SelectCharTable}{%
  % In theory, we should only detect trailing semicolons or braces,
  % but that would require un-doing the marking for any other character.
  % The next best thing is to undo the marking for closing parentheses,
  % because loops or if statements are the only places where we will
  % reasonably have a semicolon in the middle of a line, and those all
  % end with a closing parenthesis.
  \lst@DefSaveDef{41}\lstsaved@closeparen{%    handle closing parenthesis
    \lstsaved@closeparen
    \ifnum\lst@mode=\lst@Pmode    % regular processing mode (not a comment)
      \global\let\lst@eolsemicolonpending\@empty  % undo semicolon setting
    \fi%
  }%
  \lst@DefSaveDef{59}\lstsaved@semicolon{%     handle semicolon
    \lstsaved@semicolon
    \ifnum\lst@mode=\lst@Pmode    % regular processing mode (not a comment)
      \global\let\lst@eolsemicolonpending\relax
    \fi%
  }%
  \lst@DefSaveDef{125}\lstsaved@closebrace{%   handle closing brace
    \lst@eolsemicolonfalse        % do not break before a closing brace
    \lstsaved@closebrace          % might invoke \lst@NewLine
    \ifnum\lst@mode=\lst@Pmode    % regular processing mode (not a comment)
      \global\let\lst@eolsemicolonpending\relax
    \fi%
  }%
}

\makeatother


%%--------------------------------------------------
%%--------------------------------------------------
%% Macros
%!TEX root = std.tex
% Definitions and redefinitions of special commands

%%--------------------------------------------------
%% Difference markups
\definecolor{addclr}{rgb}{0,0.5,0.1}
\definecolor{remclr}{rgb}{1,0,0}
\definecolor{noteclr}{rgb}{0,0,1}

\renewcommand{\added}[1]{\textcolor{addclr}{\uline{#1}}}
\newcommand{\removed}[1]{\textcolor{remclr}{\sout{#1}}}
\renewcommand{\changed}[2]{\removed{#1}\added{#2}}

\newcommand{\nbc}[1]{[#1]\ }
\newcommand{\addednb}[2]{\added{\nbc{#1}#2}}
\newcommand{\removednb}[2]{\removed{\nbc{#1}#2}}
\newcommand{\changednb}[3]{\removednb{#1}{#2}\added{#3}}
\newcommand{\remitem}[1]{\item\removed{#1}}

\newcommand{\ednote}[1]{\textcolor{noteclr}{[Editor's note: #1] }}
% \newcommand{\ednote}[1]{}

\newenvironment{addedblock}
{
\color{addclr}
}
{
\color{black}
}
\newenvironment{removedblock}
{
\color{remclr}
}
{
\color{black}
}

%%--------------------------------------------------
% General code style
\newcommand{\CodeStyle}{\ttfamily}
\newcommand{\CodeStylex}[1]{\texttt{#1}}

% Code and definitions embedded in text.
\newcommand{\tcode}[1]{\CodeStylex{#1}}
\newcommand{\techterm}[1]{\textit{#1}}
\newcommand{\defnx}[2]{\indexdefn{#2}\textit{#1}}
\newcommand{\defn}[1]{\defnx{#1}{#1}}
\newcommand{\term}[1]{\textit{#1}}
\newcommand{\grammarterm}[1]{\textit{#1}}
\newcommand{\grammartermnc}[1]{\textit{#1}\nocorr}
\newcommand{\placeholder}[1]{\textit{#1}}
\newcommand{\placeholdernc}[1]{\textit{#1\nocorr}}

%%--------------------------------------------------
%% allow line break if needed for justification
\newcommand{\brk}{\discretionary{}{}{}}

%%--------------------------------------------------
%% Macros for funky text
\newcommand{\Cpp}{\texorpdfstring{C\kern-0.05em\protect\raisebox{.35ex}{\textsmaller[2]{+\kern-0.05em+}}}{C++}}
\newcommand{\CppIII}{\Cpp{} 2003}
\newcommand{\CppXI}{\Cpp{} 2011}
\newcommand{\CppXIV}{\Cpp{} 2014}
\newcommand{\CppXVII}{\Cpp{} 2017}
\newcommand{\opt}[1]{\ifthenelse{\equal{#1}{}}
    {\PackageError{main}{argument must not be empty}{}}
    {#1\ensuremath{_\mathit{opt}}}}
\newcommand{\dcr}{-{-}}
\newcommand{\bigoh}[1]{\ensuremath{\mathscr{O}(#1)}}

% Make all tildes a little larger to avoid visual similarity with hyphens.
\renewcommand{\~}{\textasciitilde}
\let\OldTextAsciiTilde\textasciitilde
\renewcommand{\textasciitilde}{\protect\raisebox{-0.17ex}{\larger\OldTextAsciiTilde}}
\newcommand{\caret}{\char`\^}

%%--------------------------------------------------
%% States and operators
\newcommand{\state}[2]{\tcode{#1}\ensuremath{_{#2}}}
\newcommand{\bitand}{\ensuremath{\, \mathsf{bitand} \,}}
\newcommand{\bitor}{\ensuremath{\, \mathsf{bitor} \,}}
\newcommand{\xor}{\ensuremath{\, \mathsf{xor} \,}}
\newcommand{\rightshift}{\ensuremath{\, \mathsf{rshift} \,}}
\newcommand{\leftshift}[1]{\ensuremath{\, \mathsf{lshift}_#1 \,}}

%% Notes and examples
\newcommand{\noteintro}[1]{[\,\textit{#1:}\space}
\newcommand{\noteoutro}[1]{\textit{\,---\,end #1}\,]}
\newenvironment{note}[1][Note]{\noteintro{#1}}{\noteoutro{note}\space}
\newenvironment{example}[1][Example]{\noteintro{#1}}{\noteoutro{example}\space}

%% Library function descriptions
\newcommand{\Fundescx}[1]{\textit{#1}}
\newcommand{\Fundesc}[1]{\Fundescx{#1:}\space}
\newcommand{\required}{\Fundesc{Required behavior}}
\newcommand{\requires}{\Fundesc{Requires}}
\newcommand{\effects}{\Fundesc{Effects}}
\newcommand{\postconditions}{\Fundesc{Postconditions}}
\newcommand{\returns}{\Fundesc{Returns}}
\newcommand{\throws}{\Fundesc{Throws}}
\newcommand{\default}{\Fundesc{Default behavior}}
\newcommand{\complexity}{\Fundesc{Complexity}}
\newcommand{\remarks}{\Fundesc{Remarks}}
\newcommand{\errors}{\Fundesc{Error conditions}}
\newcommand{\sync}{\Fundesc{Synchronization}}
\newcommand{\implimits}{\Fundesc{Implementation limits}}
\newcommand{\replaceable}{\Fundesc{Replaceable}}
\newcommand{\returntype}{\Fundesc{Return type}}
\newcommand{\cvalue}{\Fundesc{Value}}
\newcommand{\ctype}{\Fundesc{Type}}
\newcommand{\ctypes}{\Fundesc{Types}}
\newcommand{\dtype}{\Fundesc{Default type}}
\newcommand{\ctemplate}{\Fundesc{Class template}}
\newcommand{\templalias}{\Fundesc{Alias template}}

%% Cross reference
\newcommand{\xref}{\textsc{See also:}\space}

%% Inline parenthesized reference
\newcommand{\iref}[1]{\nolinebreak[3] (\ref{#1})}

%% NTBS, etc.
\newcommand{\NTS}[1]{\textsc{#1}}
\newcommand{\ntbs}{\NTS{ntbs}}
\newcommand{\ntmbs}{\NTS{ntmbs}}
% The following are currently unused:
% \newcommand{\ntwcs}{\NTS{ntwcs}}
% \newcommand{\ntcxvis}{\NTS{ntc16s}}
% \newcommand{\ntcxxxiis}{\NTS{ntc32s}}

%% Code annotations
\newcommand{\EXPO}[1]{\textit{#1}}
\newcommand{\expos}{\EXPO{exposition only}}
\newcommand{\impdef}{\EXPO{implementation-defined}}
\newcommand{\impdefnc}{\EXPO{implementation-defined\nocorr}}
\newcommand{\impdefx}[1]{\indeximpldef{#1}\EXPO{implementation-defined}}
\newcommand{\notdef}{\EXPO{not defined}}

\newcommand{\UNSP}[1]{\textit{\texttt{#1}}}
\newcommand{\UNSPnc}[1]{\textit{\texttt{#1}\nocorr}}
\newcommand{\unspec}{\UNSP{unspecified}}
\newcommand{\unspecnc}{\UNSPnc{unspecified}}
\newcommand{\unspecbool}{\UNSP{unspecified-bool-type}}
\newcommand{\seebelow}{\UNSP{see below}}
\newcommand{\seebelownc}{\UNSPnc{see below}}
\newcommand{\unspecuniqtype}{\UNSP{unspecified unique type}}
\newcommand{\unspecalloctype}{\UNSP{unspecified allocator type}}

\newcommand{\EXPLICIT}{\textit{\texttt{EXPLICIT}\nocorr}}

%% Manual insertion of italic corrections, for aligning in the presence
%% of the above annotations.
\newlength{\itcorrwidth}
\newlength{\itletterwidth}
\newcommand{\itcorr}[1][]{%
 \settowidth{\itcorrwidth}{\textit{x\/}}%
 \settowidth{\itletterwidth}{\textit{x\nocorr}}%
 \addtolength{\itcorrwidth}{-1\itletterwidth}%
 \makebox[#1\itcorrwidth]{}%
}

%% Double underscore
\newcommand{\ungap}{\kern.5pt}
\newcommand{\unun}{\textunderscore\ungap\textunderscore}
\newcommand{\xname}[1]{\tcode{\unun\ungap#1}}
\newcommand{\mname}[1]{\tcode{\unun\ungap#1\ungap\unun}}

%% An elided code fragment, /* ... */, that is formatted as code.
%% (By default, listings typeset comments as body text.)
%% Produces 9 output characters.
\newcommand{\commentellip}{\tcode{/* ...\ */}}

%% Ranges
\newcommand{\Range}[4]{\tcode{#1#3,\penalty2000{} #4#2}}
\newcommand{\crange}[2]{\Range{[}{]}{#1}{#2}}
\newcommand{\brange}[2]{\Range{(}{]}{#1}{#2}}
\newcommand{\orange}[2]{\Range{(}{)}{#1}{#2}}
\newcommand{\range}[2]{\Range{[}{)}{#1}{#2}}

%% Change descriptions
\newcommand{\diffdef}[1]{\hfill\break\textbf{#1:}\space}
\newcommand{\diffref}[1]{\pnum\textbf{Affected subclause:} \ref{#1}}
\newcommand{\change}{\diffdef{Change}}
\newcommand{\rationale}{\diffdef{Rationale}}
\newcommand{\effect}{\diffdef{Effect on original feature}}
\newcommand{\difficulty}{\diffdef{Difficulty of converting}}
\newcommand{\howwide}{\diffdef{How widely used}}

%% Miscellaneous
\newcommand{\uniquens}{\placeholdernc{unique}}
\newcommand{\stage}[1]{\item[Stage #1:]}
\newcommand{\doccite}[1]{\textit{#1}}
\newcommand{\cvqual}[1]{\textit{#1}}
\newcommand{\cv}{\cvqual{cv}}
\newcommand{\numconst}[1]{\textsl{#1}}
\newcommand{\logop}[1]{{\footnotesize #1}}

%%--------------------------------------------------
%% Environments for code listings.

% We use the 'listings' package, with some small customizations.  The
% most interesting customization: all TeX commands are available
% within comments.  Comments are set in italics, keywords and strings
% don't get special treatment.

\lstset{language=C++,
        basicstyle=\small\CodeStyle,
        keywordstyle=,
        stringstyle=,
        xleftmargin=1em,
        showstringspaces=false,
        commentstyle=\itshape\rmfamily,
        columns=fullflexible,
        keepspaces=true,
        texcl=true}

% Our usual abbreviation for 'listings'.  Comments are in
% italics.  Arbitrary TeX commands can be used if they're
% surrounded by @ signs.
\newcommand{\CodeBlockSetup}{
 \lstset{escapechar=@, aboveskip=\parskip, belowskip=0pt,
         midpenalty=500, endpenalty=-50,
         emptylinepenalty=-250, semicolonpenalty=0}
 \renewcommand{\tcode}[1]{\textup{\CodeStylex{##1}}}
 \renewcommand{\techterm}[1]{\textit{\CodeStylex{##1}}}
 \renewcommand{\term}[1]{\textit{##1}}
 \renewcommand{\grammarterm}[1]{\textit{##1}}
}

\lstnewenvironment{codeblock}{\CodeBlockSetup}{}

% An environment for command / program output that is not C++ code.
\lstnewenvironment{outputblock}{\lstset{language=}}{}

% A code block in which single-quotes are digit separators
% rather than character literals.
\lstnewenvironment{codeblockdigitsep}{
 \CodeBlockSetup
 \lstset{deletestring=[b]{'}}
}{}

% Permit use of '@' inside codeblock blocks (don't ask)
\makeatletter
\newcommand{\atsign}{@}
\makeatother

%%--------------------------------------------------
%% Indented text
\newenvironment{indented}[1][]
{\begin{indenthelper}[#1]\item\relax}
{\end{indenthelper}}

%%--------------------------------------------------
%% Library item descriptions
\lstnewenvironment{itemdecl}
{
 \lstset{escapechar=@,
 xleftmargin=0em,
 midpenalty=500,
 semicolonpenalty=-50,
 endpenalty=3000,
 aboveskip=2ex,
 belowskip=0ex	% leave this alone: it keeps these things out of the
				% footnote area
 }
}
{
}

\newenvironment{itemdescr}
{
 \begin{indented}[beginpenalty=3000, endpenalty=-300]}
{
 \end{indented}
}

%%--------------------------------------------------
%% add special hyphenation rules
\hyphenation{tem-plate ex-am-ple in-put-it-er-a-tor name-space name-spaces non-zero}

%%--------------------------------------------------
%% turn off all ligatures inside \texttt
\DisableLigatures{encoding = T1, family = tt*}






\newcommand{\forceindent}{\parindent=1em\indent\parindent=0pt\relax} % For indenting a paragraph containing code that can't be laid out as a {codeblock} because it also contains \emph

\begin{document}
\title{Portable assumptions}
\author{
  Timur Doumler \small(\href{mailto:papers@timur.audio}{papers@timur.audio})
}
\date{}
\maketitle

\begin{tabular}{ll}
Document \#: & P1774R4 \\
Date: & 2021-11-14\\
Project: & Programming Language C++ \\
Audience: & Evolution Working Group
\end{tabular}


\begin{abstract}
We propose a standard facility providing the semantics of existing compiler intrinsics such as \tcode{__builtin_assume} (Clang) and \tcode{__assume} (MSVC, Intel). It gives the programmer a way to allow the compiler to assume that a given C++ expression is true, without evaluating it, and to optimise based on this assumption. This is very useful for high-performance and low-latency applications in order to generate both faster and smaller code.
\end{abstract}

\vspace{5mm}

\section{Motivation}

All major compilers offer built-ins that give the programmer a way to allow the compiler to assume that a given C++ expression is true, and to optimise based on this assumption. They are very useful for high-performance and low-latency applications in order to generate both faster and smaller code. Use cases include more efficient code generation for mathematical operations, better vectorisation of loops, elision of unnecessary branches, function calls, and more.

Consider the following function:

\begin{codeblock}
int divide_by_32(int x) 
{
  __builtin_assume(x >= 0);
  return x/32;
}
\end{codeblock}

Without the assumption, the compiler has to generate code that works correctly for all possible input values. With the assumption, there is no need to generate code that handles the case of a negative numerator. The calculation can therefore be performed using a single instruction (shift right by 5 bits). Here is the output generated by clang (trunk) with \tcode{-O3}:

\begin{multicols}{2}
Without \tcode{__builtin_assume}:

\begin{codeblock}
  mov eax, edi
  sar eax, 31
  shr eax, 27
  add eax, edi
  sar eax, 5
  ret
\end{codeblock}

\columnbreak

With \tcode{__builtin_assume}:

\begin{codeblock}
  mov eax, edi
  shr eax, 5
  ret
\end{codeblock}

\end{multicols}

All major compilers offer this functionality by providing the following built-ins (see \cite{N4425} for a more thorough discussion):
\begin{itemize}
\item MSVC and Intel provide \tcode{__assume(\emph{expression});}
\item Clang provides \tcode{__builtin_assume(\emph{expression});}
\item GCC does not provide an analogous built-in. The desired effect can be achieved with \\
 \tcode{if (\emph{expression}) \{\} else  \{ __builtin_unreachable(); \}} \\
However, an important difference is that here, unlike in the other cases, \tcode{\emph{expression}} will be evaluated, at least in principle (in practice, this can often be optimised away).
\end{itemize}

Assumptions are a useful expert-level feature, and they are existing practice. However, they are spelled differently on every compiler and subject to subtle differences and ambiguities in semantics, which are not properly defined anywhere. The goal of this proposal is to introduce a unified syntax and well-defined semantics for assumptions in a way that is compatible with all existing compiler implementations and fits well into the existing C++ standard.

%%%%%%%%%%%%%%%%%%%%%%%%%%%%%%%%%%%%%%%%


\section{Syntax}


\subsection{Proposed}

We propose an attribute syntax to spell portable assumptions. \tcode{__builtin_assume(\emph{expression})} instead becomes:

\forceindent
\tcode{[[assume(\emph{expression})]]}

First of all, we propose that the word ``assume'' is used in the spelling this feature. This is the name already used in existing built-ins, therefore choosing it means standardising existing practice. This name will be least surprising and most self-explanatory to the user.

The syntax above (using parentheses) is chosen such that it is fully compatible with standard attribute syntax and therefore backwards-compatible with a compiler that does not support this feature. 

We advise against a syntax involving a colon, such as \tcode{[[assume:\phantom{x}\emph{expression}]]} or other variations that deviate from existing C++ attribute grammar, because this would require otherwise unnecessary changes to the C++ grammar, and make it harder to add assumptions to existing code.

Making this an attribute also makes it clear to the user that, given a valid C++ program that contains the attribute, ignoring it does not change the observable semantics of such a program.

It is further consistent with existing optimisation-related attributes (\tcode{[[likely]]}, \tcode{[[unlikely]]}, \tcode{[[carries_dependency]]}) as well as existing attributes that increase the space of undefined behaviour in a C++ program (\tcode{[[noreturn]]}).

We also believe that this syntax has the least impact on the core language as opposed to the alternatives (see below).

\subsection{Alternatives considered (not proposed)}

\subsubsection{Keyword}

An assumption can be characterised as an operator with an unevaluated operand, somewhat similar to \tcode{decltype(\emph{expression})}. We could consider proposing a new keyword for this new operator, such that \tcode{__builtin_assume(\emph{expression})} instead becomes:

\forceindent
\tcode{assume(\emph{expression})}

However, portable assumptions are a low-level expert feature, with the potential to inject undefined behaviour into an otherwise valid program. It should be used carefully and sparingly. We therefore advise against introducing a new keyword for this feature.


\subsubsection{Macro}

Instead of introducing a keyword, we could introduce an \tcode{assume} macro, analogous to how \tcode{assert} is already defined as a macro. However, macros are known to cause many problems. Their lack of scoping can lead to name clashes, the preprocessor grammar makes it impossible to use curly braces inside the expression, etc. For these and other reasons, modern C++ tries to minimise the use of macros. We therefore advise against introducing a new macro for this feature.

\subsubsection{``Magic'' library function}

Alternatively, we could introduce portable assumptions in the form of a ``magic'' library function, so \tcode{__builtin_assume(\emph{expression})} instead becomes:

\forceindent
 \tcode{std::assume(\emph{expression});}

However, such a spelling would introduce a weird novelty into the C++ language: something that is syntactically a function call, yet does not evaluate its operand. It would essentially be something like a ``namespaced keyword'', and very different in nature to all existing ``magic'' library functions. Significant core language changes would be needed to make it work. We also believe that such a construct would be surprising to C++ developers. We therefore advise against adding this feature as a library function.

A seeming advantage is the consistency with the closely related \tcode{std::assume_aligned} \cite{P1007R3}, which was adopted for C++20. However, they are different. For \tcode{std::assume_aligned},  unlike for an assumption, the operand may be evaluated, so the problem above does not arise for \tcode{std::assume_aligned} (or, in fact, any other current ``magic'' library function).



\section{Semantics}

The design goal is to provide a portable facility closely following the compiler built-ins \tcode{__assume} and \tcode{__builtin_assume}, therefore standardising existing practice. The facility should be implementable with the existing built-ins on those compiler implementations who have them, without unnecessarily constraining implementations who do not. Therefore, we propose the following semantics:
\begin{itemize}
\item It is a statement with a single argument, which is a C++ expression contextually convertible to \tcode{bool}.
\item The expression is an unevaluated operand, like for example the operand of \tcode{decltype}. Therefore, expressions with side effects are allowed (which is useful, consider \tcode{++ptr != end}). However, as the expression is not evaluated, these side effects do not affect the behaviour of the program\footnote{However, just like with \tcode{decltype}, the expression might cause a template instantiation if this is required to analyse the expression. Thanks to Nathan Sidwell for pointing this out.}.
\item However, the optimiser may analyse the form of the expression, and deduce from that information used to optimise the program.
\item The behaviour is undefined if the expression would \emph{not} evaluate to \tcode{true}\footnote{Note that there is a subtle difference between behaviour being undefined if the expression would evaluate to \tcode{false}, or if the expression would \emph{not} evaluate to \tcode{true}. The latter (proposed here) also includes the assumption that the expression itself would not result in undefined behaviour if it were evaluated. This enlarges the space of assumptions that can be stated by the programmer. Thanks to Joshua Berne for pointing this out.}. This allows the optimiser to optimise the program based on the assumption that it always will.
\item Simply ignoring the whole statement is a conforming implementation, i.e. the optimiser is not required in any way to make use of the assumption.
\end{itemize}

%%%%%%%%%%%%%%%%%%%%%%%%%%%%%%%%

\section{Benchmarks}

Since assumptions are not standardised yet, for this study we use an \tcode{ASSUME} macro instead of the syntax proposed above. The macro is defined as follows and works on all major compilers:

\begin{codeblock}
#if defined(__clang__)
  #define ASSUME(expr) __builtin_assume(expr)
#elif defined(__GNUC__) && !defined(__ICC)
  #define ASSUME(expr) if (expr) {} else { __builtin_unreachable(); }
#elif defined(_MSC_VER) || defined(__ICC)
  #define ASSUME(expr) __assume(expr)
#endif
\end{codeblock}

As discussed above, there are subtle differences in semantics across compilers; however, these differences have no relevance for the benchmarks presented here.

\subsection{Code size}

TODO

Another example: consider looping over an array of numbers and performing math on the elements. Often, there are invariants on the array size such as: it's a power of two, it's a multiple of the SIMD register size, etc (all very common e.g. in audio processing code). Telling the optimiser about such invariants leads to a much better optimisation and vectorisation of the loop:

\begin{codeblock}
void limiter(float* buffer, size_t size) 
{
  __builtin_assume(size % 8 == 0);
  for (size_t i = 0; i < size; ++i)
     data[i] = std::clamp(data[i], -1.0f, 1.0f);
}
\end{codeblock}

For this function, clang (trunk) with \tcode{-O3} generates 70 lines of assembly without the assumption, and only 42 lines with it.

See \cite{Regehr2014} for more examples and use cases. 

\subsection{Execution time}

TODO

%%%%%%%%%%%%%%%%%%%%%%%%%%%%%%%%%%%%%%%%

\section{History and related work}

\subsection{N4425 and Contracts}

Adding such portable optimisation hints was already proposed once \cite{N4425} and discussed by EWG in 2015 in Lenexa\footnote{\url{https://cplusplus.github.io/EWG/ewg-closed.html\#179}}. The paper was rejected. EWG's guidance was that this functionality should be provided within the proposed contracts facility, and not as a separate feature.

Unfortunately, contracts as merged into the C++20 working draft in June 2018 in Rapperswil \cite{P0542R5}, actually failed to provide such portable optimisation hints \cite{P1773R0}. Later, in July 2019 in Cologne, contracts were pulled from C++20 altogether.

Regardless of whether contracts will eventually make it into a future C++ standard, and whether or not assumptions might be a feature of such contracts, we need an independent low-level ``assume'' facility.  The present paper focuses on proposing exactly this low-level facility. Its purpose is to introduce a standard way to provide an optimisation hint to the compiler, as an implementation detail of a C++ program, locally, with clearly defined semantics that are expressed in C++ code, independent of any build modes, build flags, etc.

In case contracts or other higher-level features will make use of such assumptions in the future, it can then be implemented in terms of this low-level facility.

\subsection{PXXXX: \tcode{std::unreachable}}

TODO

\subsection{P2064: Herb Sutter's Assumptions paper}

TODO

\section {Previous polls}

Below are the polls taken by WG21 subgroups on previous revisions of this paper.

\subsection{EWG, Prague (February 2020)}

1. We want assumptions now and independent of future contract facilities.

\hspace{6mm}
\begin{tabular}{lllll}
SF & F & N & A & SA \\
18 & 5 & 1 & 3 & 3
\end{tabular}

2. We like the proposed semantics for assumptions.

\hspace{6mm}
\begin{tabular}{lllll}
SF & F & N & A & SA \\
18 & 5 & 4 & 2 & 0
\end{tabular}

3. We want exploration on a mode which can check assumptions, including side effects.

\hspace{6mm}
\begin{tabular}{lllll}
SF & F & N & A & SA \\
1 & 0 & 9 & 9 & 5
\end{tabular}

4. We like the proposed attribute syntax \tcode{[[assume(expr)]]}

\hspace{6mm}
\begin{tabular}{lllll}
SF & F & N & A & SA \\
9 & 8 & 5 & 5 & 1
\end{tabular}

5. We’d like more exploration on macro assume, like assert

\hspace{6mm}
\begin{tabular}{lllll}
SF & F & N & A & SA \\
0 & 0 & 1 & 10 & 16
\end{tabular}

6. We’d like more exploration on keyword such as one of \tcode{unsafe_assume} / \tcode{assume} / \tcode{__assume} / \tcode{_Assume} / …

\hspace{6mm}
\begin{tabular}{lllll}
SF & F & N & A & SA \\
5 & 7 & 9 & 5 & 2
\end{tabular}

7. We’d like more exploration on magic library function such as \tcode{std::assume(expr)}.

\hspace{6mm}
\begin{tabular}{lllll}
SF & F & N & A & SA \\
0 & 0 & 0 & 9 & 14
\end{tabular}

\subsection{SG21, Prague (February 2020)}

Assumptions should proceed independently of contracts.

\hspace{6mm}
\begin{tabular}{lllll}
SF & F & N & A & SA \\
9 & 8 & 5 & 6 & 5
\end{tabular}


\subsection{EWG, Belfast (November 2019)}

1.. P1774 with \tcode{[[assume(expr)]]} syntax.

\hspace{6mm}
\begin{tabular}{lllll}
SF & F & N & A & SA \\
15 & 5 & 1 & 0 & 0
\end{tabular}

2. P1774 with \tcode{std::assume(expr)} syntax.

\hspace{6mm}
\begin{tabular}{lllll}
SF & F & N & A & SA \\
1 & 3 & 4 & 10 & 4
\end{tabular}

\section{Proposed wording}

Add the following subclause to [dcl.attr]:

\begin{adjustwidth}{0.5cm}{0.5cm}
\begin{addedblock}
\subsection*{Assumption attribute \hspace{7.33cm} [dcl.attr.assume]}

The \textit{attribute-token} \tcode{assume} may be applied to a null statement; such a statement is an assumption. The attribute-token \tcode{assume} shall appear at most once in each attribute-list. An \textit{attribute-argument-clause} shall be present and shall have the form:

\hspace{5mm}\tcode{( }\textit{assignment-expression}\tcode{ )}

The epression shall be contextually convertible to \tcode{bool}. The expression is not evaluated, however, it is potentially evaluated [basic.ref.odr]. If the expression would not evaluate to \tcode{true}, the behavior is undefined.


\begin{note}
The use of assumptions is intended to allow implementations to analyze the form of the expression and deduce from that information used to optimize the program.
\end{note}

\begin{example}
\begin{codeblock}
int divide_by_32(int x)  {
    [[assume(x >= 0)]];
    return x/32;   // The instructions produced for the division
                   //  may omit handling of negative values
}
\end{codeblock}
\end{example}
\end{addedblock}
\end{adjustwidth}

Modify [expr.const] as follows:

\begin{adjustwidth}{0.5cm}{0.5cm}
If \tcode{e} satisfies the constraints of a core constant expression, but evaluation of e would evaluate an operation that has undefined behavior as specified in [library] through [thread] of this document, \added{a statement with an assumption ([dcl.attr.assume]) that has a predicate that would not evaluate to \tcode{true}, }or an invocation of the \tcode{va_­start} macro ([cstdarg.syn]), it is unspecified whether \tcode{e} is a core constant expression.
\end{adjustwidth}

\section*{Document history}

\begin{itemize}
\item \textbf{R0}, 2019-06-17: Initial version.
\item \textbf{R1}, 2019-10-06: Updated text to reflect removal of Contracts from C++20; made proposed attribute syntax backwards-compatible by replacing colon with parentheses.
\item \textbf{R2}, 2019-11-25: Changed title to ``Portable assumptions''; changed semantics from UB if expression would evaluate to \tcode{false} to UB if expression would \emph{not} evaluate to \tcode{true}; changed syntax section to propose attribute-syntax only, dropping ``magic'' library function syntax as a viable alternative.
\item \textbf {R3}, 2020-01-13: Updated text to clarify the discussion of the proposed semantics and syntax.
\item \textbf{R4}, 2021-11-14:  Added wording. Clarified that the assumed expression is unevaluated, but odr-used. Added polls. Added code size measurement results. Updated and re-structured text; added discussion of recent related work.
\end{itemize}

\section*{Acknowledgements}

Many thanks to Herb Sutter, Chandler Carruth, Michael Spencer, Jonathan Caves, and Hal Finkel for their very helpful comments and explanations.

\renewcommand{\bibname}{References}
\bibliographystyle{abstract}
\bibliography{ref}

\end{document}