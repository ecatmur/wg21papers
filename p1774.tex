%% Common header for WG21 proposals ? mainly taken from C++ standard draft source
%%

%%--------------------------------------------------
%% basics
\documentclass[a4paper,11pt,oneside,openany,final,article]{memoir}

\usepackage[american]
           {babel}        % needed for iso dates
\usepackage[iso,american]
           {isodate}      % use iso format for dates
\usepackage[final]
           {listings}     % code listings
\usepackage{longtable}    % auto-breaking tables
\usepackage{ltcaption}    % fix captions for long tables
\usepackage{relsize}      % provide relative font size changes
\usepackage{textcomp}     % provide \text{l,r}angle
\usepackage{underscore}   % remove special status of '_' in ordinary text
\usepackage{parskip}      % handle non-indented paragraphs "properly"
\usepackage{array}        % new column definitions for tables
\usepackage[normalem]{ulem}
\usepackage{enumitem}
\usepackage{color}        % define colors for strikeouts and underlines
\usepackage{xcolor}    % needed for blue links

\usepackage{amsmath}      % additional math symbols
\usepackage{mathrsfs}     % mathscr font
\usepackage[final]{microtype}
\usepackage{multicol}
\usepackage{lmodern}
\usepackage[T1]{fontenc}
\usepackage[pdftex, final]{graphicx}
\usepackage[pdftex,
            bookmarks=true,
            bookmarksnumbered=true,
            pdfpagelabels=true,
            pdfpagemode=UseOutlines,
            pdfstartview=FitH,
            linktocpage=true,
            colorlinks=true,
            plainpages=false,
            allcolors={blue}, 
            allbordercolors={white}
           ]{hyperref}
\usepackage{memhfixc}     % fix interactions between hyperref and memoir
\usepackage{url}  % urls in ref.bib
\usepackage{tabularx}  % don't use the C++ standard's fancy tables, they come with captions!

\pdfminorversion=5
\pdfcompresslevel=9
\pdfobjcompresslevel=2

\renewcommand\RSsmallest{5.5pt}  % smallest font size for relsize

%%--------------------------------------------------
%%--------------------------------------------------
%% Layout -- set overall page appearance

%%--------------------------------------------------
%%  set page size, type block size, type block position

\setlrmarginsandblock{2.245cm}{2.245cm}{*}
\setulmarginsandblock{2.5cm}{2.5cm}{*}

%%--------------------------------------------------
%%  set header and footer positions and sizes

\setheadfoot{\onelineskip}{2\onelineskip}
\setheaderspaces{*}{2\onelineskip}{*}

%%--------------------------------------------------
%%  make miscellaneous adjustments, then finish the layout
\setmarginnotes{7pt}{7pt}{0pt}
\checkandfixthelayout

%%--------------------------------------------------
%% If there is insufficient stretchable vertical space on a page,
%% TeX will not properly consider penalties for a good page break,
%% even if \raggedbottom (default) is in effect.
\addtolength{\topskip}{0pt plus 20pt}

%%--------------------------------------------------
%% Paragraph and bullet numbering

\newcounter{Paras}
\counterwithout{section}{chapter}
\setcounter{secnumdepth}{3}

\newcounter{Bullets1}[Paras]
\newcounter{Bullets2}[Bullets1]
\newcounter{Bullets3}[Bullets2]
\newcounter{Bullets4}[Bullets3]

\makeatletter
\newcommand{\parabullnum}[2]{%
\stepcounter{#1}%
\noindent\makebox[0pt][l]{\makebox[#2][r]{%
\scriptsize\raisebox{.7ex}%
{%
\ifnum \value{Paras}>0
\ifnum \value{Bullets1}>0 (\fi%
                          \arabic{Paras}%
\ifnum \value{Bullets1}>0 .\arabic{Bullets1}%
\ifnum \value{Bullets2}>0 .\arabic{Bullets2}%
\ifnum \value{Bullets3}>0 .\arabic{Bullets3}%
\fi\fi\fi%
\ifnum \value{Bullets1}>0 )\fi%
\fi%
}%
\hspace{\@totalleftmargin}\quad%
}}}
\makeatother

\def\pnum{\parabullnum{Paras}{0pt}}

%%--------------------------------------------------
%%--------------------------------------------------
%% Styles
%!TEX root = std.tex
%% styles.tex -- set styles for:
%     chapters
%     pages
%     footnotes

%%--------------------------------------------------
%%  create chapter style

\makechapterstyle{cppstd}{%
  \renewcommand{\beforechapskip}{\onelineskip}
  \renewcommand{\afterchapskip}{\onelineskip}
  \renewcommand{\chapternamenum}{}
  \renewcommand{\chapnamefont}{\chaptitlefont}
  \renewcommand{\chapnumfont}{\chaptitlefont}
  \renewcommand{\printchapternum}{\chapnumfont\thechapter\quad}
  \renewcommand{\afterchapternum}{}
}

%%--------------------------------------------------
%%  create page styles




%%--------------------------------------------------
% set style for main text
\setlength{\parindent}{0pt}
\setlength{\parskip}{1ex}

%%--------------------------------------------------
%% change list item markers to number and em-dash

\renewcommand{\labelitemi}{---\parabullnum{Bullets1}{\labelsep}}
\renewcommand{\labelitemii}{---\parabullnum{Bullets2}{\labelsep}}
\renewcommand{\labelitemiii}{---\parabullnum{Bullets3}{\labelsep}}
\renewcommand{\labelitemiv}{---\parabullnum{Bullets4}{\labelsep}}



%%--------------------------------------------------
%% override some functions from the listings package to avoid bad page breaks
%% (copied verbatim from listings.sty version 1.6 except where commented)
\makeatletter

\def\lst@Init#1{%
    \begingroup
    \ifx\lst@float\relax\else
        \edef\@tempa{\noexpand\lst@beginfloat{lstlisting}[\lst@float]}%
        \expandafter\@tempa
    \fi
    \ifx\lst@multicols\@empty\else
        \edef\lst@next{\noexpand\multicols{\lst@multicols}}
        \expandafter\lst@next
    \fi
    \ifhmode\ifinner \lst@boxtrue \fi\fi
    \lst@ifbox
        \lsthk@BoxUnsafe
        \hbox to\z@\bgroup
             $\if t\lst@boxpos \vtop
        \else \if b\lst@boxpos \vbox
        \else \vcenter \fi\fi
        \bgroup \par\noindent
    \else
        \lst@ifdisplaystyle
            \lst@EveryDisplay
            % make penalty configurable
            \par\lst@beginpenalty
            \vspace\lst@aboveskip
        \fi
    \fi
    \normalbaselines
    \abovecaptionskip\lst@abovecaption\relax
    \belowcaptionskip\lst@belowcaption\relax
    \lst@MakeCaption t%
    \lsthk@PreInit \lsthk@Init
    \lst@ifdisplaystyle
        \global\let\lst@ltxlabel\@empty
        \if@inlabel
            \lst@ifresetmargins
                \leavevmode
            \else
                \xdef\lst@ltxlabel{\the\everypar}%
                \lst@AddTo\lst@ltxlabel{%
                    \global\let\lst@ltxlabel\@empty
                    \everypar{\lsthk@EveryLine\lsthk@EveryPar}}%
            \fi
        \fi
        % A section heading might have set \everypar to apply a \clubpenalty
        % to the following paragraph, changing \everypar in the process.
        % Unconditionally overriding \everypar is a bad idea.
        % \everypar\expandafter{\lst@ltxlabel
        %                      \lsthk@EveryLine\lsthk@EveryPar}%
    \else
        \everypar{}\let\lst@NewLine\@empty
    \fi
    \lsthk@InitVars \lsthk@InitVarsBOL
    \lst@Let{13}\lst@MProcessListing
    \let\lst@Backslash#1%
    \lst@EnterMode{\lst@Pmode}{\lst@SelectCharTable}%
    \lst@InitFinalize}

\def\lst@DeInit{%
    \lst@XPrintToken \lst@EOLUpdate
    \global\advance\lst@newlines\m@ne
    \lst@ifshowlines
        \lst@DoNewLines
    \else
        \setbox\@tempboxa\vbox{\lst@DoNewLines}%
    \fi
    \lst@ifdisplaystyle \par\removelastskip \fi
    \lsthk@ExitVars\everypar{}\lsthk@DeInit\normalbaselines\normalcolor
    \lst@MakeCaption b%
    \lst@ifbox
        \egroup $\hss \egroup
        \vrule\@width\lst@maxwidth\@height\z@\@depth\z@
    \else
        \lst@ifdisplaystyle
            % make penalty configurable
            \par\lst@endpenalty
            \vspace\lst@belowskip
        \fi
    \fi
    \ifx\lst@multicols\@empty\else
        \def\lst@next{\global\let\@checkend\@gobble
                      \endmulticols
                      \global\let\@checkend\lst@@checkend}
        \expandafter\lst@next
    \fi
    \ifx\lst@float\relax\else
        \expandafter\lst@endfloat
    \fi
    \endgroup}


\def\lst@NewLine{%
    \ifx\lst@OutputBox\@gobble\else
        \par
        % add configurable penalties
        \lst@ifeolsemicolon
          \lst@semicolonpenalty
          \lst@eolsemicolonfalse
        \else
          \lst@domidpenalty
        \fi
        % Manually apply EveryLine and EveryPar; do not depend on \everypar
        \noindent \hbox{}\lsthk@EveryLine%
        % \lsthk@EveryPar uses \refstepcounter which balloons the PDF
    \fi
    \global\advance\lst@newlines\m@ne
    \lst@newlinetrue}

% new macro for empty lines, avoiding an \hbox that cannot be discarded
\def\lst@DoEmptyLine{%
  \ifvmode\else\par\fi\lst@emptylinepenalty
  \vskip\parskip
  \vskip\baselineskip
  % \lsthk@EveryLine has \lst@parshape, i.e. \parshape, which causes an \hbox
  % \lsthk@EveryPar increments line counters; \refstepcounter balloons the PDF
  \global\advance\lst@newlines\m@ne
  \lst@newlinetrue}

\def\lst@DoNewLines{
    \@whilenum\lst@newlines>\lst@maxempty \do
        {\lst@ifpreservenumber
            \lsthk@OnEmptyLine
            \global\advance\c@lstnumber\lst@advancelstnum
         \fi
         \global\advance\lst@newlines\m@ne}%
    \@whilenum \lst@newlines>\@ne \do
        % special-case empty printing of lines
        {\lsthk@OnEmptyLine\lst@DoEmptyLine}%
    \ifnum\lst@newlines>\z@ \lst@NewLine \fi}

% add keys for configuring before/end vertical penalties
\lst@Key{beginpenalty}\relax{\def\lst@beginpenalty{\penalty #1}}
\let\lst@beginpenalty\@empty
\lst@Key{midpenalty}\relax{\def\lst@midpenalty{\penalty #1}}
\let\lst@midpenalty\@empty
\lst@Key{endpenalty}\relax{\def\lst@endpenalty{\penalty #1}}
\let\lst@endpenalty\@empty
\lst@Key{emptylinepenalty}\relax{\def\lst@emptylinepenalty{\penalty #1}}
\let\lst@emptylinepenalty\@empty
\lst@Key{semicolonpenalty}\relax{\def\lst@semicolonpenalty{\penalty #1}}
\let\lst@semicolonpenalty\@empty

\lst@AddToHook{InitVars}{\let\lst@domidpenalty\@empty}
\lst@AddToHook{InitVarsEOL}{\let\lst@domidpenalty\lst@midpenalty}

% handle semicolons and closing braces (could be in \lstdefinelanguage as well)
\def\lst@eolsemicolontrue{\global\let\lst@ifeolsemicolon\iftrue}
\def\lst@eolsemicolonfalse{\global\let\lst@ifeolsemicolon\iffalse}
\lst@AddToHook{InitVars}{
  \global\let\lst@eolsemicolonpending\@empty
  \lst@eolsemicolonfalse
}
% If we found a semicolon or closing brace while parsing the current line,
% inform the subsequent \lst@NewLine about it for penalties.
\lst@AddToHook{InitVarsEOL}{%
  \ifx\lst@eolsemicolonpending\relax
    \lst@eolsemicolontrue
    \global\let\lst@eolsemicolonpending\@empty
  \fi%
}
\lst@AddToHook{SelectCharTable}{%
  % In theory, we should only detect trailing semicolons or braces,
  % but that would require un-doing the marking for any other character.
  % The next best thing is to undo the marking for closing parentheses,
  % because loops or if statements are the only places where we will
  % reasonably have a semicolon in the middle of a line, and those all
  % end with a closing parenthesis.
  \lst@DefSaveDef{41}\lstsaved@closeparen{%    handle closing parenthesis
    \lstsaved@closeparen
    \ifnum\lst@mode=\lst@Pmode    % regular processing mode (not a comment)
      \global\let\lst@eolsemicolonpending\@empty  % undo semicolon setting
    \fi%
  }%
  \lst@DefSaveDef{59}\lstsaved@semicolon{%     handle semicolon
    \lstsaved@semicolon
    \ifnum\lst@mode=\lst@Pmode    % regular processing mode (not a comment)
      \global\let\lst@eolsemicolonpending\relax
    \fi%
  }%
  \lst@DefSaveDef{125}\lstsaved@closebrace{%   handle closing brace
    \lst@eolsemicolonfalse        % do not break before a closing brace
    \lstsaved@closebrace          % might invoke \lst@NewLine
    \ifnum\lst@mode=\lst@Pmode    % regular processing mode (not a comment)
      \global\let\lst@eolsemicolonpending\relax
    \fi%
  }%
}

\makeatother


%%--------------------------------------------------
%%--------------------------------------------------
%% Macros
%!TEX root = std.tex
% Definitions and redefinitions of special commands

%%--------------------------------------------------
%% Difference markups
\definecolor{addclr}{rgb}{0,0.5,0.1}
\definecolor{remclr}{rgb}{1,0,0}
\definecolor{noteclr}{rgb}{0,0,1}

\renewcommand{\added}[1]{\textcolor{addclr}{\uline{#1}}}
\newcommand{\removed}[1]{\textcolor{remclr}{\sout{#1}}}
\renewcommand{\changed}[2]{\removed{#1}\added{#2}}

\newcommand{\nbc}[1]{[#1]\ }
\newcommand{\addednb}[2]{\added{\nbc{#1}#2}}
\newcommand{\removednb}[2]{\removed{\nbc{#1}#2}}
\newcommand{\changednb}[3]{\removednb{#1}{#2}\added{#3}}
\newcommand{\remitem}[1]{\item\removed{#1}}

\newcommand{\ednote}[1]{\textcolor{noteclr}{[Editor's note: #1] }}
% \newcommand{\ednote}[1]{}

\newenvironment{addedblock}
{
\color{addclr}
}
{
\color{black}
}
\newenvironment{removedblock}
{
\color{remclr}
}
{
\color{black}
}

%%--------------------------------------------------
% General code style
\newcommand{\CodeStyle}{\ttfamily}
\newcommand{\CodeStylex}[1]{\texttt{#1}}

% Code and definitions embedded in text.
\newcommand{\tcode}[1]{\CodeStylex{#1}}
\newcommand{\techterm}[1]{\textit{#1}}
\newcommand{\defnx}[2]{\indexdefn{#2}\textit{#1}}
\newcommand{\defn}[1]{\defnx{#1}{#1}}
\newcommand{\term}[1]{\textit{#1}}
\newcommand{\grammarterm}[1]{\textit{#1}}
\newcommand{\grammartermnc}[1]{\textit{#1}\nocorr}
\newcommand{\placeholder}[1]{\textit{#1}}
\newcommand{\placeholdernc}[1]{\textit{#1\nocorr}}

%%--------------------------------------------------
%% allow line break if needed for justification
\newcommand{\brk}{\discretionary{}{}{}}

%%--------------------------------------------------
%% Macros for funky text
\newcommand{\Cpp}{\texorpdfstring{C\kern-0.05em\protect\raisebox{.35ex}{\textsmaller[2]{+\kern-0.05em+}}}{C++}}
\newcommand{\CppIII}{\Cpp{} 2003}
\newcommand{\CppXI}{\Cpp{} 2011}
\newcommand{\CppXIV}{\Cpp{} 2014}
\newcommand{\CppXVII}{\Cpp{} 2017}
\newcommand{\opt}[1]{\ifthenelse{\equal{#1}{}}
    {\PackageError{main}{argument must not be empty}{}}
    {#1\ensuremath{_\mathit{opt}}}}
\newcommand{\dcr}{-{-}}
\newcommand{\bigoh}[1]{\ensuremath{\mathscr{O}(#1)}}

% Make all tildes a little larger to avoid visual similarity with hyphens.
\renewcommand{\~}{\textasciitilde}
\let\OldTextAsciiTilde\textasciitilde
\renewcommand{\textasciitilde}{\protect\raisebox{-0.17ex}{\larger\OldTextAsciiTilde}}
\newcommand{\caret}{\char`\^}

%%--------------------------------------------------
%% States and operators
\newcommand{\state}[2]{\tcode{#1}\ensuremath{_{#2}}}
\newcommand{\bitand}{\ensuremath{\, \mathsf{bitand} \,}}
\newcommand{\bitor}{\ensuremath{\, \mathsf{bitor} \,}}
\newcommand{\xor}{\ensuremath{\, \mathsf{xor} \,}}
\newcommand{\rightshift}{\ensuremath{\, \mathsf{rshift} \,}}
\newcommand{\leftshift}[1]{\ensuremath{\, \mathsf{lshift}_#1 \,}}

%% Notes and examples
\newcommand{\noteintro}[1]{[\,\textit{#1:}\space}
\newcommand{\noteoutro}[1]{\textit{\,---\,end #1}\,]}
\newenvironment{note}[1][Note]{\noteintro{#1}}{\noteoutro{note}\space}
\newenvironment{example}[1][Example]{\noteintro{#1}}{\noteoutro{example}\space}

%% Library function descriptions
\newcommand{\Fundescx}[1]{\textit{#1}}
\newcommand{\Fundesc}[1]{\Fundescx{#1:}\space}
\newcommand{\required}{\Fundesc{Required behavior}}
\newcommand{\requires}{\Fundesc{Requires}}
\newcommand{\effects}{\Fundesc{Effects}}
\newcommand{\postconditions}{\Fundesc{Postconditions}}
\newcommand{\returns}{\Fundesc{Returns}}
\newcommand{\throws}{\Fundesc{Throws}}
\newcommand{\default}{\Fundesc{Default behavior}}
\newcommand{\complexity}{\Fundesc{Complexity}}
\newcommand{\remarks}{\Fundesc{Remarks}}
\newcommand{\errors}{\Fundesc{Error conditions}}
\newcommand{\sync}{\Fundesc{Synchronization}}
\newcommand{\implimits}{\Fundesc{Implementation limits}}
\newcommand{\replaceable}{\Fundesc{Replaceable}}
\newcommand{\returntype}{\Fundesc{Return type}}
\newcommand{\cvalue}{\Fundesc{Value}}
\newcommand{\ctype}{\Fundesc{Type}}
\newcommand{\ctypes}{\Fundesc{Types}}
\newcommand{\dtype}{\Fundesc{Default type}}
\newcommand{\ctemplate}{\Fundesc{Class template}}
\newcommand{\templalias}{\Fundesc{Alias template}}

%% Cross reference
\newcommand{\xref}{\textsc{See also:}\space}

%% Inline parenthesized reference
\newcommand{\iref}[1]{\nolinebreak[3] (\ref{#1})}

%% NTBS, etc.
\newcommand{\NTS}[1]{\textsc{#1}}
\newcommand{\ntbs}{\NTS{ntbs}}
\newcommand{\ntmbs}{\NTS{ntmbs}}
% The following are currently unused:
% \newcommand{\ntwcs}{\NTS{ntwcs}}
% \newcommand{\ntcxvis}{\NTS{ntc16s}}
% \newcommand{\ntcxxxiis}{\NTS{ntc32s}}

%% Code annotations
\newcommand{\EXPO}[1]{\textit{#1}}
\newcommand{\expos}{\EXPO{exposition only}}
\newcommand{\impdef}{\EXPO{implementation-defined}}
\newcommand{\impdefnc}{\EXPO{implementation-defined\nocorr}}
\newcommand{\impdefx}[1]{\indeximpldef{#1}\EXPO{implementation-defined}}
\newcommand{\notdef}{\EXPO{not defined}}

\newcommand{\UNSP}[1]{\textit{\texttt{#1}}}
\newcommand{\UNSPnc}[1]{\textit{\texttt{#1}\nocorr}}
\newcommand{\unspec}{\UNSP{unspecified}}
\newcommand{\unspecnc}{\UNSPnc{unspecified}}
\newcommand{\unspecbool}{\UNSP{unspecified-bool-type}}
\newcommand{\seebelow}{\UNSP{see below}}
\newcommand{\seebelownc}{\UNSPnc{see below}}
\newcommand{\unspecuniqtype}{\UNSP{unspecified unique type}}
\newcommand{\unspecalloctype}{\UNSP{unspecified allocator type}}

\newcommand{\EXPLICIT}{\textit{\texttt{EXPLICIT}\nocorr}}

%% Manual insertion of italic corrections, for aligning in the presence
%% of the above annotations.
\newlength{\itcorrwidth}
\newlength{\itletterwidth}
\newcommand{\itcorr}[1][]{%
 \settowidth{\itcorrwidth}{\textit{x\/}}%
 \settowidth{\itletterwidth}{\textit{x\nocorr}}%
 \addtolength{\itcorrwidth}{-1\itletterwidth}%
 \makebox[#1\itcorrwidth]{}%
}

%% Double underscore
\newcommand{\ungap}{\kern.5pt}
\newcommand{\unun}{\textunderscore\ungap\textunderscore}
\newcommand{\xname}[1]{\tcode{\unun\ungap#1}}
\newcommand{\mname}[1]{\tcode{\unun\ungap#1\ungap\unun}}

%% An elided code fragment, /* ... */, that is formatted as code.
%% (By default, listings typeset comments as body text.)
%% Produces 9 output characters.
\newcommand{\commentellip}{\tcode{/* ...\ */}}

%% Ranges
\newcommand{\Range}[4]{\tcode{#1#3,\penalty2000{} #4#2}}
\newcommand{\crange}[2]{\Range{[}{]}{#1}{#2}}
\newcommand{\brange}[2]{\Range{(}{]}{#1}{#2}}
\newcommand{\orange}[2]{\Range{(}{)}{#1}{#2}}
\newcommand{\range}[2]{\Range{[}{)}{#1}{#2}}

%% Change descriptions
\newcommand{\diffdef}[1]{\hfill\break\textbf{#1:}\space}
\newcommand{\diffref}[1]{\pnum\textbf{Affected subclause:} \ref{#1}}
\newcommand{\change}{\diffdef{Change}}
\newcommand{\rationale}{\diffdef{Rationale}}
\newcommand{\effect}{\diffdef{Effect on original feature}}
\newcommand{\difficulty}{\diffdef{Difficulty of converting}}
\newcommand{\howwide}{\diffdef{How widely used}}

%% Miscellaneous
\newcommand{\uniquens}{\placeholdernc{unique}}
\newcommand{\stage}[1]{\item[Stage #1:]}
\newcommand{\doccite}[1]{\textit{#1}}
\newcommand{\cvqual}[1]{\textit{#1}}
\newcommand{\cv}{\cvqual{cv}}
\newcommand{\numconst}[1]{\textsl{#1}}
\newcommand{\logop}[1]{{\footnotesize #1}}

%%--------------------------------------------------
%% Environments for code listings.

% We use the 'listings' package, with some small customizations.  The
% most interesting customization: all TeX commands are available
% within comments.  Comments are set in italics, keywords and strings
% don't get special treatment.

\lstset{language=C++,
        basicstyle=\small\CodeStyle,
        keywordstyle=,
        stringstyle=,
        xleftmargin=1em,
        showstringspaces=false,
        commentstyle=\itshape\rmfamily,
        columns=fullflexible,
        keepspaces=true,
        texcl=true}

% Our usual abbreviation for 'listings'.  Comments are in
% italics.  Arbitrary TeX commands can be used if they're
% surrounded by @ signs.
\newcommand{\CodeBlockSetup}{
 \lstset{escapechar=@, aboveskip=\parskip, belowskip=0pt,
         midpenalty=500, endpenalty=-50,
         emptylinepenalty=-250, semicolonpenalty=0}
 \renewcommand{\tcode}[1]{\textup{\CodeStylex{##1}}}
 \renewcommand{\techterm}[1]{\textit{\CodeStylex{##1}}}
 \renewcommand{\term}[1]{\textit{##1}}
 \renewcommand{\grammarterm}[1]{\textit{##1}}
}

\lstnewenvironment{codeblock}{\CodeBlockSetup}{}

% An environment for command / program output that is not C++ code.
\lstnewenvironment{outputblock}{\lstset{language=}}{}

% A code block in which single-quotes are digit separators
% rather than character literals.
\lstnewenvironment{codeblockdigitsep}{
 \CodeBlockSetup
 \lstset{deletestring=[b]{'}}
}{}

% Permit use of '@' inside codeblock blocks (don't ask)
\makeatletter
\newcommand{\atsign}{@}
\makeatother

%%--------------------------------------------------
%% Indented text
\newenvironment{indented}[1][]
{\begin{indenthelper}[#1]\item\relax}
{\end{indenthelper}}

%%--------------------------------------------------
%% Library item descriptions
\lstnewenvironment{itemdecl}
{
 \lstset{escapechar=@,
 xleftmargin=0em,
 midpenalty=500,
 semicolonpenalty=-50,
 endpenalty=3000,
 aboveskip=2ex,
 belowskip=0ex	% leave this alone: it keeps these things out of the
				% footnote area
 }
}
{
}

\newenvironment{itemdescr}
{
 \begin{indented}[beginpenalty=3000, endpenalty=-300]}
{
 \end{indented}
}

%%--------------------------------------------------
%% add special hyphenation rules
\hyphenation{tem-plate ex-am-ple in-put-it-er-a-tor name-space name-spaces non-zero}

%%--------------------------------------------------
%% turn off all ligatures inside \texttt
\DisableLigatures{encoding = T1, family = tt*}






\newcommand{\forceindent}{\parindent=1em\indent\parindent=0pt\relax} % For indenting a paragraph containing code that can't be laid out as a {codeblock} because it also contains \emph

\begin{document}
\title{Portable assumptions}
\author{
  Timur Doumler \small(\href{mailto:papers@timur.audio}{papers@timur.audio})
}
\date{}
\maketitle

\begin{tabular}{ll}
Document \#: & D1774R5 \\
Date: & 2021-12-02\\
Project: & Programming Language C++ \\
Audience: & Evolution Working Group, Core Working Group
\end{tabular}


\begin{abstract}
We propose a standard facility providing the semantics of existing compiler intrinsics such as \tcode{__builtin_assume} (Clang) and \tcode{__assume} (MSVC, ICC). It gives the programmer a way to allow the compiler to assume that a given C++ expression is true, without evaluating it, and to optimise based on this assumption. This is very useful for high-performance and low-latency applications in order to generate both faster and smaller code.
\end{abstract}

\vspace{5mm}

\section{Motivation}

All major compilers offer built-ins that give the programmer a way to allow the compiler to assume that a given C++ expression is true, and to optimise based on this assumption. They are very useful for high-performance and low-latency applications in order to generate both faster and smaller code. Use cases include more efficient code generation for mathematical operations, better vectorisation of loops, elision of unnecessary branches, function calls, and more.

Consider the following function (from \cite{Regehr2014}):

\begin{codeblock}
int divide_by_32(int x) 
{
  __builtin_assume(x >= 0);
  return x/32;
}
\end{codeblock}

Without the assumption, the compiler has to generate code that works correctly for all possible input values. With the assumption, there is no need to generate code that handles the case of a negative numerator. The calculation can therefore be performed using a single instruction (shift right by 5 bits). Here is the output generated by clang (trunk) with \tcode{-O3}:

\begin{multicols}{2}
Without \tcode{__builtin_assume}:

\begin{codeblock}
  mov eax, edi
  sar eax, 31
  shr eax, 27
  add eax, edi
  sar eax, 5
  ret
\end{codeblock}

\columnbreak

With \tcode{__builtin_assume}:

\begin{codeblock}
  mov eax, edi
  shr eax, 5
  ret
\end{codeblock}

\end{multicols}

All major compilers offer this functionality by providing the following built-ins (see \cite{N4425} and Section \ref{sec:semantics} below for a more thorough discussion):
\begin{itemize}
\item MSVC and ICC have \tcode{__assume(\emph{expr});}
\item Clang has \tcode{__builtin_assume(\emph{expr});}
\item GCC has \tcode{__builtin_unreachable()}, which works slightly differently. Assuming an expression can be written as \tcode{if (\emph{expr}) \{\} else  \{ __builtin_unreachable(); \}} \\ 
However, this has slightly different semantics: on GCC, \tcode{\emph{expr}} will be evaluated, at least notionally. In practice, this will often be optimised away if evaluating \tcode{\emph{expr}} has no side effects. 
\end{itemize}

Assumptions are a useful expert-level feature, and they are existing practice. However, they are spelled differently on every compiler and subject to subtle differences and ambiguities in semantics, which are not properly defined anywhere. The goal of this proposal is to introduce a unified syntax and well-defined semantics for assumptions in a way that is compatible with all existing compiler implementations and fits well into the existing C++ standard.

%%%%%%%%%%%%%%%%%%%%%%%%%%%%%%%%%%%%%%%%


\section{Syntax}


\subsection{Proposed}

We propose an attribute syntax to spell portable assumptions:

\forceindent
\tcode{[[assume(\emph{expr})]]}

First of all, we propose that the word ``assume'' is used in the spelling this feature. This is the name already used in existing built-ins, therefore choosing it means standardising existing practice. This name will be least surprising and most self-explanatory to the user.

The syntax above (using parentheses) is chosen such that it is fully compatible with standard attribute syntax and therefore backwards-compatible with a compiler that does not support this feature. 

Making this an attribute also makes it clear that assumptions share an important property with the other C++ attributes: given a valid C++ program that contains the attribute, ignoring it does not change the observable semantics of such a program.

It is further consistent with existing optimisation-related attributes (\tcode{[[likely]]}, \tcode{[[unlikely]]}, \tcode{[[carries_dependency]]}) as well as existing attributes that increase the space of undefined behaviour in a C++ program (\tcode{[[noreturn]]}). More generally, attributes tend to target the back-end of the compiler and/or other tools in the C++ ecosystem, rather than the front-end. This is true for assumptions as well, which are targeting the optimiser. Therefore, assumptions should be an attribute.

We further believe that this syntax has the least impact on the existing core language as opposed to the alternatives discussed below.

Finally, the attribute syntax would also allow to add this feature to the C language with the same spelling.

Herb Sutter argues in \cite{P2064R0} against this attribute syntax, saying that it would ``make assumes awkward to write in the one place they should appear, which is a statement'', and that it ``would allow asuumes to be written outside of function bodies''. Neither of these are true. We specify \tcode{[[assume(expr)]]} to be an attribute that can only be applied to a null statement, just like we already do with \tcode{[[fallthrough]]}. The effect of this is that it can only appear on its own, as a statement, followed by a semicolon, and only inside a function body, which is exactly the intended use.

\subsection{Alternatives considered (not proposed)}
\label{subsec:syntax_alternatives}

\subsubsection{New syntax}

We explored syntax involving a colon, such as \tcode{[[assume:\phantom{x}\emph{expression}]]}, the syntax used in \cite{P0542R5} for contracts, and other variations that deviate from existing C++ attribute grammar. 

We do not see any benefit of introducing a new syntax to C++ over using existing attribute syntax. New syntax would require otherwise unnecessary changes to the C++ grammar, making it harder to add assumptions to existing code due to lack of backwards-compatibility. At the same time, it does not provide any benefits over the attribute syntax we propose.

In addition, using syntax too similar to that used by contracts proposals is actively harmful: assumptions are a feature completely separate from contracts (see Section \ref{sec:contracts}), and the syntax should therefore be separate from contracts as well.

\subsubsection{Keyword}

An assumption could be described as an operator, somewhat similar to \tcode{decltype(\emph{expr})}, where \tcode{decltype} is a keyword and \emph{expr} is an unevaluated operand. We therefore considered to add a new keyword for assumptions, so that the spelling becomes:

\forceindent
\tcode{assume(\emph{expression})}

We could also spell such a keyword differently. \cite{P2064R0} suggests the spelling \tcode{unsafe_assume} to highlight that this is a narrow, low-level, expert-only feature, with the potential to inject undefined behaviour into an otherwise valid program, and should therefore be used with great care.

However, for exactly this reason, we believe that a new keyword is not the right approach. Adding a new keyword is a very significant change to the language. A narrow, expert-only feature that will only be used by a small fraction of developers does not justify a new keyword.

\subsubsection{Macro}

Instead of introducing a keyword, we could introduce an \tcode{assume} macro, analogous to how \tcode{assert} is already defined as a macro (and again, we could spell it in different ways). However, macros are known to cause many problems. Their lack of scoping can lead to name clashes, the preprocessor grammar makes it impossible to use curly braces inside the expression, etc. For these and other reasons, modern C++ tries to minimise the use of macros. We don't see any good reason to deviate from this principle.

\subsubsection{``Magic'' library function}

One option that at first glance seems very attractive is to spell an assumption as a ``magic'' library function:

\forceindent
 \tcode{std::assume(\emph{expression});}
 
Herb Sutter \cite{P2064R0} and others have argued for such an approach. However, a deeper analysis reveals that this is not a viable route. Making assumptions a function would introduce a weird novelty into the C++ language: something that is syntactically a function call, yet does not evaluate its argument. This would be very different in nature to all existing ``magic'' library functions. Apart from not evaluating the argument of the function call, such a function would differ from other C++ language functions in many other ways. It would look like a standard C++ function, but it would behave like built-ins such as \tcode{__builtin_assume} behave today: the only thing that you can do with them is to directly call them. You can't take their address, you can’t assign them to a function pointer, etc. By making assumptions a function, we would essentially be saying that it's a function but it's so special that the only properties it shares with an actual function is that it has a name and an argument list. It would effectively be a namespaced keyword.

Significant core language changes would be needed to make such a novelty work, adding more complexity to a fundamental part of the core language (what is a function call?). We do not believe that assumptions come anywhere close to justifying such changes to the language. The proposed attribute syntax avoids all this complexity by using a mechanism that already exists in the language.

It has been pointed out that the spelling \tcode{std::assume} would be consistent with the related \tcode{std::assume_aligned}, which was adopted for C++20. However, as should be clear from the above discussion, they are fundamentally different. For \tcode{std::assume_aligned},  unlike for an assumption, the argument may be evaluated, just like for any other function call in C++. The problem described above does therefore not arise for \tcode{std::assume_aligned} (or any other existing ``magic'' library function in C++).



\section{Proposed semantics}
\label{sec:semantics}

We collaborated with compiler engineers from MSVC, GCC, Clang, ICC, and EDG, to make sure that the semantics proposed here for standard C++ are implementable on all these compilers and are compatible with the de-facto semantics of all the existing assumption built-ins. We also incorporated feedback from all previous rounds of EWG review as well as discussions on the WG21 reflectors.

\subsection{Constraints on the argument}

The argument clause of an \tcode{assume} attribute must be present and must contain a single expression contextually convertible to \tcode{bool}. It has been pointed out that this allows to write non-sensical assumptions such as \tcode{[[assume("Hello")]]}. However, this is fully consistent with existing assumption built-ins as well as the rest of the C++ language. Carving out exceptions to prevent users from writing certain types of non-sensical assumptions would add unnecessary complexity for little benefit.

\subsection{The expression is not evaluated}

The expression inside an assumption is unevaluated, like for example the operand of \tcode{decltype}. This is a fundamental property of assumptions. The purpose is to assume the expression without checking it (see also section \ref{sec:contracts}, and Herb Sutter's discussion in \cite{P2064R0} why assumptions are fundamentally different from assertions).

Expressions with side effects are allowed inside an assumption, but any such side effects will not be executed and will not affect the behaviour of the program. This is consistent with both the semantics of attributes in C++ and the semantics of existing \tcode{__assume} and \tcode{__builtin_assume} (for an in-depth discussion of assumptions with side effects, see section \ref{subsec:side_effects}).

GCC is currently the only major compiler that doesn't have an intrinsic with the unevaluated expression semantics. In GCC, we have to spell assumptions like this:

\begin{codeblock}
if (expr) {} else { __builtin_unreachable(); }
\end{codeblock}

which evaluates \tcode{expr}. However, GCC can implement assume in a way conforming to the semantics proposed here relatively easily, by employing the following strategy. First, it can check whether \tcode{expr} can have side effects if evaluated (fGCC has a facility for this). If it can prove that it cannot, it can express the assumption in terms of its existing \tcode{__builtin_unreachable()}. Otherwise, it can simply ignore the assumption. Ignoring assumptions with potential side effects is a conforming implementation of the semantics proposed here.

Ignoring assumptions altogether is also a conforming implementation. A trivial implementation of assumptions is therefore possible on any C++ compiler. The only requirement is that the assumed expression is checked for well-formedness (see section \ref{subsec:well_formed}).

\subsection{Assumptions that would not evaluate to \tcode{true} are undefined behaviour}

The argument of an assumption is not evaluated, however the optimiser may analyse it, and deduce information from that analysis that it can use to optimise the program. The crucial property of an assumption is that if such analysis reveals that it would evaluate to \tcode{true}, the assumption has no effect, but otherwise the behaviour is undefined. This gives the compiler the freedom to optimise away any code path that could be reached if the assumption is not \tcode{true}. This includes so-called ``time travel'' optimisation. Consider the following function (example from \cite{P2064R0}:

\begin{codeblock}
int f(int j) {
  int i = 42;
  if (j == 0)
    i = 0;

  [[assume(j != 0)]];
  return i;
}
\end{codeblock}

The proposed semantics allow the optimiser to assume that \tcode{j != 0} was already true before the code reached the assumption, since \tcode{j} was not modified. It can therefore remove the branch before the assumption, and reduce the whole function to \tcode{return 42}. This is merely specifying existing practice: both GCC and Clang actually perform this optimisation.

There is a subtle difference between behaviour being undefined if the expression would evaluate to \tcode{false}, or if the expression would \emph{not} evaluate to \tcode{true}. The latter (proposed here) also includes the assumption that the expression would actually return a value, not throw an exception, and not exhibit undefined behaviour if it were evaluated. This enlarges the space of assumptions that can be stated by the programmer (thanks to Joshua Berne for pointing this out). Undefined behaviour inside the assumed expression is therefore allowed to escape the assumption, even if the expression is notionally unevaluated.

\subsection{Assumptions ODR-use their argument}

At first glance, this requirement seems unnecessary. If the argument of an assumption is not evaluated, why would we want to specify that it is ODR-used? ODR-use means that, even if the assumption is otherwise ignored, the assumed expression can trigger template instantiations.

The answer is: ODR-use is what existing implementations do. Implementing assumptions without ODR-use of the argument would be extremely difficult. As far as we know, such an implementation does not exist.

MSVC, ICC, and Clang all follow the same basic principle to implement assumptions. The compiler actually generates intermediate representation for the expression inside the assumption (which requires ODR-use). This code is then used during optimisation of the program. At a later stage of the optimiser, the assumption-related code is then stripped out again (the exact mechanics of this vary from compiler to compiler).

\subsection{Semantics of side effects}
\label{subsec:side_effects}

Assuming expressions with side effects is occasionally useful (consider \tcode{[[assume(++ptr != end)]]}). MSVC, Clang, and ICC all allow to write such assumptions, and at least MSVC uses them for some optimisations. But at first glance, it does not seem obvious how to formally define the semantics of such an assumption in terms of the C++ abstract machine. If \tcode{ptr} is not actually incremented at the point where the assumption occurs, how can we reason about a counterfactual world in which \tcode{ptr} is incremented, and make assumptions about the program (in which \tcode{ptr} is not incremented at that point) based on that? It seems that we would need to introduce some novel concept of ``hypothetical evaluation'' of an unevaluated expression in the standard, requiring herculean efforts.

As we will show below, in fact no such thing is needed to understand the semantics of assumptions. We begin by categorising all expressions that could appear in an assumption into three categories:

\begin{itemize}
\item Category 1. The assumed expression has no side effects when evaluated.
\item Category 2. The assumed expression may have side effects, but they are deterministic.
\item Category 3. The assumed expression may have non-deterministic side effects.
\end{itemize}

Let us now discuss the semantics of each category.

\textbf{Category 1.} This is the most common type of assumptions and the most straightforward. Consider the following minimal example:

\begin{codeblock}
int f(int i) {
  [[assume(i == 42)]];
  return i;
}
\end{codeblock}

The implementation can assume that \tcode{i == 42} evaluates to \tcode{true}, and optimise based on this assumption. The evaluation of this expression has no side effects, therefore it doesn't actually matter if the expression is evaluated: any instructions emitted for such an evaluation won't affect the observable behaviour of the program and can be optimised away afterwards. This is the only category of assumptions that can also be expressed by GCC's \tcode{__builtin_unerachable()}. An implementation is allowed to either ignore the assumption, or optimise \tcode{f} as follows:

\begin{codeblock}
int f(int i) {
  return 42;
}
\end{codeblock}

\textbf{Category 2.} Let us now consider the following minimal example:

\begin{codeblock}
int f(int i) {
  [[assume(++i == 43)]];
  return i;
}
\end{codeblock}

This assumption has a side effect: evaluating the expression would modify \tcode{i}. The implementation is not allowed to do this. However, it is allowed to ``analyse the form of the expression and deduce information used to optimise the program''. Note that since the semantics of integer increment are known and deterministic, the statement \tcode{++i == 43}, which would have side effects if evaluated, can be transformed to an equivalent statement that does \emph{not} have side effects if evaluated: \tcode{i == 42}. Assuming this new statement is equivalent to assuming the original statement. This program is therefore equivalent to the one in the previous example.

In other words, the statement about a hypothetical program in which \tcode{i} would be incremented can be reduced to a statement about the real program, in which \tcode{i} is not being incremented, at the point where the assumption occurs. The resulting statement is a Category 1 assumption, which has well-defined semantics. Since we know that the side effects are deterministic, such a reduction to a Category 1 assumption is always possible, at least theoretically.

Of course, real-world compilers won't be able to perform the required transformation in all cases. This is not a problem, since a compiler is allowed to just ignore the assumption if it cannot derive any useful information from it.

\textbf{Category 3.} Let us now consider the following pathological example (from Martin Uecker and Aaron Ballman):

\begin{codeblock}
int f(int i) {
  [[assume((std::cin >> i, i == 42))]];
  std::cin >> i;
  return i;
}
\end{codeblock}

Of course, nobody should ever write such an assumption, as it obviously does not express an invariant of the program and therefore cannot serve a useful purpose. But nevertheless we need to be able to determine what assumption this code expresses and what semantics it has. The crucial difference to a Category 2 assumption is that the input value received from \tcode{std::cin} is non-deterministic, yet it is required to determine whether the assumption holds and the program is well-defined. The expression can only be evaluated by calling \tcode{std::cin}, but we are not allowed to do that, since the argument of the assumption is unevaluated.

At first glance, this seems like a paradox, and various contradicting interpretations seem possible, including:
\begin{itemize}
\item There is no useful information that can be derived from this assumption, therefore it should have no effect. The compiler must translate the program as written ignoring the assumption.
\item The compiler cannot actually call \tcode{std::cin} inside the assumption, since assumptions are unevaluated. It is therefore impossible to determine what the value of the expression would be. Since it does not ``evaluate to \tcode{true}'', the program is undefined behaviour, and the compiler is allowed to optimise out the whole function \tcode{f} and all code paths leading to it.
\item The compiler can assume that a call to \tcode{std::cin} at the point of the assumption would read the number \tcode{42}. Since there is no change in program state between this point and the point where \tcode{std::cin} is actually called (on the next line), the compiler is allowed to optimise out the call to \tcode{std::cin} and replace the code with \tcode{int f(int) \{ return 42; \}}
\end{itemize}

It seems that we cannot answer which of these interpretations is correct without specifying the semantics further, in particular without specifying the meaning of expressions that would have to be evaluated, yet are not evaluated. However, as it turns out, it is actually straightforward to reason about assumptions, using the specification in this proposal, as soon as we give up this idea of ``hypothetical evaluation''. This is fundamentally the wrong mental model to reason about assumptions. It then turns out that all three of the above interpretations are incorrect. The correct reasoning goes as follows.

First of all, note that we do not need to consider the behaviour on any system where the above assumption does not hold, since the specification does not put any constraints on the behaviour of such a system. Therefore, we only need to consider systems where the assumption \emph{does} hold, i.e. where \tcode{std::cin}, if executed at the place where the assumption appears, would always read in the number 42. This can be, for example, a computer controlled by a robot which is programmed to always enter the number 42 when prompted. On this system, the assumption is doing exactly what it is intended to do: it expresses an invariant of the system (the robot will always type 42) which the C++ compiler cannot see (as it is unaware of the robot).

Now, on such a system where the assumption holds, its semantics are precisely defined: it has no effect. The program therefore \emph{must} call \tcode{std::cin}, as this is an observable side effect; the compiler is not allowed to optimise out the call. However, under the as-if rule, it is allowed to throw out the value read by \tcode{std::cin}, as it knows that it will always be 42. Therefore, the compiler can either leave \tcode{f} as is, or replace it with
\begin{codeblock}
int f(int) {
  int tmp;
  std::cin >> tmp;
  return 42;
}
\end{codeblock}

In other words, any Category 3 assumption (i.e. an assumption containing a non-deterministic expression) can be reduced to a Category 2 assumption by considering only systems where the assumption expresses an actual, real invariant of the program (therefore, the expression is not actually non-deterministic). On such systems, the assumption has no effect, making the semantics of the program well-defined. On all other systems, the behaviour is undefined.

To give yet another example of assumptions with side effects, let us consider the following code (from Ga\v sper A\v zman):

\begin{codeblock}
int f(ForwardIterator auto almost_last, ForwardIterator auto last) {
  [[assume(++almost_last == last)]];
  // do something... 
}
\end{codeblock}

Let us start by categorising this assumption as above. The first question is whether incrementing the forward iterator and then comparing it to the other iterator is deterministic.

If ForwardIterator is e.g. \tcode{std::forward_list<int>::iterator}, the expression is deterministic and the assumption is therefore in Category 2. Compared to our previous Category 2 example \tcode{[[assume(++i  == 43)]]}, we now cannot derive an equivalent side-effect-free equality expression like \tcode{[[assume(i == 42)]]} by reversing the increment, because \tcode{operator++} on a ForwardIterator is not reversible. However, we are never incrementing it in the first place, as the assumed expression is not evaluated (not even ``hypothetically evaluated''), only analysed. We could therefore perform the reduction to Category 1 by transforming the assumption into a side-effect-free statement about \tcode{almost_last} and its internals, such as: if we dereference \tcode{almost_last->next} and then inspect its value, it will have some property in the current program at the point where the assumption is made. It doesn’t matter if \tcode{f} isn’t allowed to access the private member variable \tcode{.next}, because the expression is not being evaluated, only analysed, and the implementation can analyse whatever it chooses.

If on the other hand, ForwardIterator is e.g. \tcode{std::istream_iterator<char>}, then the behaviour is non-deterministic, we are in Category 3, and we can apply the same reasoning as in the \tcode{std::cin} example.

It is important to remember that the above reasoning only serves to understand the semantics of expression side effects inside assumptions as defined in the proposed wording. In practice, the compiler is allowed to use a completely different strategy, including simply discarding the assumption, as long as it is compatible with these semantics.

\subsection{Behaviour of assumptions during constant evaluation}

What should happen if an assumption is encountered during constant evaluation? This is unlikely to occur in practice, since assumptions are inherently a run-time utility, but for completeness' sake we need to specify this. Consider the following code:

\begin{codeblock}
constexpr int f() {
  return 0;
}

constexpr int g() {
  [[assume(f() == 1)]];  // assumption doesn't hold
  return 1;
}

int main() {
  return g();
}
\end{codeblock}

We propose that, if such an assumption would not evaluate to \tcode{true}, it is implementation-defined whether the program is ill-formed or not. This way, we leave freedom for implementations to conduct such an analysis at compile time and emit a compiler error for a failed assumption (which can be useful), while not requiring an implementation to do so (because it might be difficult to implement for all cases, and currently none of MSVC, GCC, or Clang implement this check with \tcode{__assume} and \tcode{__builtin_assume}, respectively: the code above passes on all of them).

If an assumption holds during constant evaluation, this should have no effect.

Another subtlety is the question what should happen if inside a \tcode{constexpr} function we encounter an assumption that would evaluate to \tcode{true}, but can not be evaluated during constant evaluation? Currently, there is implementation divergence. MSVC rejects the following code with \tcode{__assume}, while GCC and Clang accept it with \tcode{__builtin_assume}:

\begin{codeblock}
int foo() {     // not a constexpr function
  return 0; 
}

constexpr int bar() {
  [[assume(foo() == 0)]];  // this assumption holds but isn't constexpr
  return 1;
}

int main() {
  return bar();
}
\end{codeblock}

We propose that this code should be well-formed. If an assumption cannot be checked at compile time, the assumption should simply be ignored, rather than making the whole program ill-formed. Otherwise, in order to be able to make the function \tcode{constexpr}, the user would have to branch on \tcode{std::is_constant_evaluated()} just for the purpose of using such an assumption.

\subsection{Ill-formed expressions need to be diagnosed}
\label{subsec:well_formed}

The C++ standard specifies that attributes not recognised by an implementation can be ignored. However, this does not extend to attributes that are part of the C++ standard itself. For the latter, the standard imposes constraints on both the argument clause of the attribute (e.g. \tcode{[[noreturn]]} must have none, \tcode{[[deprecated]]} can have one but it must be a string literal) and what entities the attribute may appertain to. If these constraints are violated, the program is ill-formed and the compiler must issue a diagnostic.

The same applies to assumptions as proposed here. A conforming compiler doesn't have to implement an assumption facility, and is free to  ignore a well-formed assumption. However, if the \tcode{assume} attribute is written in the wrong place, or doesn't have an expression as its argument, or the expression is not contextually convertible to \tcode{bool} or ill-formed, the compiler must detect this and issue a diagnostic. Further, an assumed expression is ODR-used, which can trigger template instantiations. If any of these instantiations makes the program ill-formed, for example by containing a \tcode{static_assert} that does not evaluate to \tcode{true}, this needs to be diagnosed as well.

A concern about this was voiced by Gabriel Dos Reis. The MSVC compiler currently does not parse the argument of a standard attribute, but treats it as token soup. This approach works well as long as the argument of the attribute is either absent or just a string literal (which is the case for the attributes that currently exist in C++20), but breaks down as soon as the attribute contains something more complex like an expression, which needs to be parsed in order to check for well-formedness. Dos Reis suggested that therefore, assumptions should not use attribute syntax.

First of all, we do not believe that there is any other syntax for assumptions that is viable (see Section \ref{subsec:syntax_alternatives}) and could get consensus at present. But most importantly, the issue is in no way specific to this proposal, but in fact concerns the design space of C++ attributes in general. The grammar for C++ attributes explicitly allows expressions as arguments. In fact, it allows any balanced token sequence and says that each attribute can define its own constraints on what arguments it accepts. The standard does not say that this should be limited to arguments that do not need parsing, and to the best of our knowledge such a limitation was never intended. In fact, both GCC and Clang are capable of parsing expressions inside attributes, and both use this for existing functionality (e.g. \tcode{gnu} attributes require parsing expressions). In addition, third-party vendors including the OpenMP specification (\cite{OpenMP5.1} section 2.1, ``Directive Format'') also use attribute argument clauses that require parsing. Furthermore, in WG21 there are other proposals for standard C++ currently in flight that use expressions inside attributes, such as \tcode{[[trivially_relocatable(expr)]]} \cite{P1144R5}. 

In conclusion, unless we want to cut off a significant segment of C++ design space and ignore existing practice and ongoing work in this area, a conforming C++ implementation would be well advised to support parsing C++ inside an attribute, and there seems to be no fundamental implementability issue with this.

\subsection{No top-level commas inside an assumption}

The proposed specification uses \emph{assignment-expression} for the \emph{attribute-argument-clause} of an assumption, rather than the top-level \emph{expression} grammar production. This has the effect that top-level commas are not allowed.

There are three reasons for this. First, if we were to allow writing \tcode{[[assume(\emph{expr1}, \emph{expr2})]]}, a user might erroneously read this as ``\tcode{\emph{expr1}} and \tcode{\emph{expr2}} are both assumed'', whereas in reality, only \tcode{\emph{expr2}} is assumed.

Second, what \tcode{[[assume(\emph{expr1}, \emph{expr2})]]} is actually saying is ``assume \tcode{\emph{expr2}} after \tcode{\emph{expr1}} has been evaluated just for its side effects''. Since assumed expressions are not actually evaluated, reasoning about side effects can get confusing (see discussion in section \ref{subsec:side_effects}) and such assumptions should be used with special care. It is therefore preferable to make this more explicit and more difficult to spell by requiring an extra pair of parentheses.

Third, there is no consistent existing practice to allow top-level commas. Clang forbids them in its \tcode{__builtin_assume}, while ICC and MSVC allow them in their \tcode{__assume}. However, according to Gabriel Dos Reis, ``the `acceptance' by MSVC is a parser accident -- don't use it as existing practice to standardise'', and we fully agree with this advice.

\subsection{Pack expansion}

The grammar for C++ attributes allows an attribute to be followed by an ellipsis. [dcl.attr.grammar] specifies: ``In an attribute-list, an ellipsis may appear only if that attribute's specification permits it. An attribute followed by an ellipsis is a pack expansion.''

 We could therefore hypothetically permit the \tcode{assume} attribute to directly support pack expansion:

\begin{codeblock}
template <int... args>
void f() {
    [[assume(args >= 0)...]];
}
\end{codeblock}

However, we do not propose this. It would require substantial additional work for a very rare usecase. Note that this can instead be expressed with a fold expression, which is equivalent to the above and works out of the box without any extra effort:

\begin{codeblock}
template <int... args>
void f() {
    [[assume(((args >= 0) && ...))]];
}
\end{codeblock}

%%%%%%%%%%%%%%%%%%%%%%%%%%%%%%%%

\section{Benchmarks}

Since assumptions are not standardised yet, for this study we use an \tcode{ASSUME} macro instead of the syntax proposed above. The macro is defined as follows and works on all major compilers:

\begin{codeblock}
#if defined(__clang__)
  #define ASSUME(expr) __builtin_assume(expr)
#elif defined(__GNUC__) && !defined(__ICC)
  #define ASSUME(expr) if (expr) {} else { __builtin_unreachable(); }
#elif defined(_MSC_VER) || defined(__ICC)
  #define ASSUME(expr) __assume(expr)
#endif
\end{codeblock}

As discussed above, there are differences in semantics across compilers, in particular whether or not \tcode{expr} is evaluated; however, by assuming only expressions without side effects we can make sure this has no impact on the benchmarks.

\subsection{Code size}

Consider looping over a range of \tcode{float}s and clamping all values between -1 and 1. This is an operation that often occurs in audio processing and is known as a \emph{limiter}:

\begin{codeblock}
void limiter(float* data, size_t size) 
{   
    for (size_t i = 0; i < size; ++i)
        data[i] = std::clamp(data[i], -1.0f, 1.0f);
}
\end{codeblock}

Often, such data has invariants which are guaranteed by the surrounding code, but this information is invisible to the optimiser, for example because the code is too complex for the optimiser to see through, or because there is a TU boundary in between. We can inject such invariants via assumptions. In this example, we inject the knowledge that data buffers contain at least 32 frames and the buffer size is a multiple of 32 (a common scenario in audio processing), and that the data does not contain NaNs or infinity:

\begin{codeblock}
void limiter(float* data, size_t size) 
{
    ASSUME(size > 0);
    ASSUME(size % 32 == 0);
    
    for (size_t i = 0; i < size; ++i) {
        ASSUME(std::isfinite(data[i]));
        data[i] = std::clamp(data[i], -1.0f, 1.0f);
    }
}
\end{codeblock}

We have compiled both versions of this function on godbolt.org with MSVC, GCC, Clang, and ICC, respectively. We have used the latest trunk versions of these compilers at the time of writing, and compiled the code with the maximum optimisation setting (\tcode{-O3} and \tcode{/O2}, respectively). As a crude estimate of code size, we counted the number of instructions emitted, with the following results.

\begin{table}[h!]
%\caption{Nonlinear Model Results} % title of Table
\centering % used for centering table
\begin{tabular}{c c c} % centered columns (4 columns)
%\hline\hline %inserts double horizontal lines
Compiler & Without \tcode{ASSUME} & With \tcode{ASSUME}  \\ [0.5ex] % inserts table
%heading
\hline % inserts single horizontal line
MSVC & 88 & 52  \\ % inserting body of the table
GCC & 77 & 42  \\
Clang & 74 & 20  \\
ICC & 62 & 62  \\
%\hline %inserts single line
\end{tabular}
%\label{table:nonlin} % is used to refer this table in the text
\end{table}

On all compilers except ICC, using \tcode{ASSUME} leads to significantly less code being emitted in this example. The injected assumptions lead to a better optimisation of the loop as well as elimination of unnecessary code inside \tcode{std::clamp}.
\subsection{Execution time}

Benchmarks comparing execution time of the same code compiled with and without assumptions are planned for the next revision of this paper.

%%%%%%%%%%%%%%%%%%%%%%%%%%%%%%%%%%%%%%%%

\section{History and related work}

\subsection{N4425 and pre-C++20 contracts proposals}
\label{sec:contracts}

Adding portable assumptions was already proposed in \cite{N4425}\footnote{The syntax proposed then was different: \tcode{true(\emph{expr})} and \tcode{false(\emph{expr})}, but the semantics were essentially the same as in this proposal.} and discussed by EWG in 2015 in Lenexa\footnote{\url{https://cplusplus.github.io/EWG/ewg-closed.html\#179}}. The paper was rejected. EWG's guidance was that this functionality should be provided within the proposed contracts facility, and not as a separate feature.

Ironically, contracts as merged into the C++20 working draft in June 2018 in Rapperswil \cite{P0542R5}, actually failed to provide the functionality of assumptions \cite{P1773R0}. And later, in July 2019 in Cologne, contracts were pulled from C++20 altogether. Progress on assumptions had been blocked for no good reason at all.

\subsection{Current work on contracts}

More recent proposals for adding contracts to C++  \cite{P2388R4}, \cite{P2461R1}, \cite{P2487R0} no longer include the possibility to assume contracts for purposes of optimisation. It will take time to develop these more recent contracts proposals until they can be added to the standard. Assumptions are useful, well-understood, existing practice, and we should standardise them now, rather than waiting for progress on contracts.

Contracts and assumptions are very different features. The purpose of contracts is to find and avoid bugs, and to document pre- and postconditions in code; they are meant to be used at API boundaries; they are primarily targeting the front-end of the compiler (or a static analyser); and they are a ``cross-cutting'' feature that is meant to be used widely throughout a codebase by many developers. By contrast, the purpose of assumptions is to make specific invariants of your code visible to the optimiser; they are meant to be an implementation detail; they are primarily targeting the back-end of the compiler; and they are a ``local'' feature that will only be used rarely, at specific locations in performance bottlenecks, and by experts only.

Further, the expressions that are typical for assumptions tend to look very different from the ones typically found in contracts. Assumptions are practically always either statements about a \tcode{bool}, or very simple mathematical expressions involving a single number or pointer. By contrast, contract preconditions and postconditions can contain significantly more complicated statements about the program, even including lambdas.

Standardising the existing practice of a low-level assumptions facility that is independent of contracts is not closing off future work. In case contracts or other higher-level features will incorporate assumptions in some form in the future, this can be specified and implemented using the low-level facility proposed here as a building block.

\subsection{Assertions vs. assumptions}

Assertions (whether as a subset of contracts or as a standalone feature) and assumptions are fundamentally different in nature. We are not aware of any study that could conclusively show that there is a measurable performance benefit from turning assertions into assumptions throughout a codebase. \cite{P2064R0} found that it actually degrades performance, while \cite{Amini2021} found that it makes no statistically significant difference at all. There are cases where injecting (correct) assumptions actually degrades performance, for example they can disable (rather than enable) the vectorisation of loops in some cases. This is similar to other C++ features interacting with the optimiser, such as \tcode{[[likely]]} and \tcode{[[unlikely]]}.

Therefore, we should not combine assertions and assumptions in the same language feature, we should make the syntax of assertions look different from assumptions, and we should especially not introduce a generic way to assume assertions. Instead, we should use assumptions explicitly in the few cases where it provably matters for performance. Assertions, on the other hand, should be a ``safe-to-use'' feature that primarily exists to find and avoid bugs. They should not be able to degrade performance of optimised code or inject undefined behaviour and ``time travel'' into an otherwise valid program.

For a much more detailed discussion of assertions vs. assumptions, see \cite{P2064R0}.

\subsection{\tcode{std::unreachable}}

\cite{P0627R6} is a related paper proposing a function \tcode{std::unreachable()}, standardising GCC's \tcode{__builtin_unreachable()} (see above): a function that has undefined behaviour when called, and therefore can be used to mark unreachable code paths.

It is important to recognise that the functionality provided by \tcode{std::unreachable()} is a strict subset of the functionality provided by assumptions as proposed here. \tcode{std::unreachable()} has the exact same semantics as \tcode{[[assume(false)]]}. Assuming an expression without side effects can be expressed with either \tcode{assume} or \tcode{std::unreachable} (although the latter is significantly more verbose), while assuming an expression with side effects can only be expressed with \tcode{assume}. Therefore, \tcode{assume} is the more general feature, and the one that should be standardised first.

That being said, the possibility to spell \tcode{[[assume(false)]]} as \tcode{std::unreachable} might still be desirable. If 
what the user wants to do is to mark unreachable control flow (unreachable branches, unreachable switch cases etc.), for example to avoid compiler warnings, then the spelling \tcode{std::unreachable} better communicates that intent. We therefore do not see a problem with both features coexisting.

\section {Previous polls}

Below are the polls taken by WG21 subgroups on previous revisions of this paper.


\subsection{EWG, Belfast (November 2019)}

1.. P1774 with \tcode{[[assume(expr)]]} syntax.

\hspace{6mm}
\begin{tabular}{lllll}
SF & F & N & A & SA \\
15 & 5 & 1 & 0 & 0
\end{tabular}

2. P1774 with \tcode{std::assume(expr)} syntax.

\hspace{6mm}
\begin{tabular}{lllll}
SF & F & N & A & SA \\
1 & 3 & 4 & 10 & 4
\end{tabular}

\subsection{EWG, Prague (February 2020)}

1. We want assumptions now and independent of future contract facilities.

\hspace{6mm}
\begin{tabular}{lllll}
SF & F & N & A & SA \\
18 & 5 & 1 & 3 & 3
\end{tabular}

2. We like the proposed semantics for assumptions.

\hspace{6mm}
\begin{tabular}{lllll}
SF & F & N & A & SA \\
18 & 5 & 4 & 2 & 0
\end{tabular}

3. We want exploration on a mode which can check assumptions, including side effects.

\hspace{6mm}
\begin{tabular}{lllll}
SF & F & N & A & SA \\
1 & 0 & 9 & 9 & 5
\end{tabular}

4. We like the proposed attribute syntax \tcode{[[assume(expr)]]}

\hspace{6mm}
\begin{tabular}{lllll}
SF & F & N & A & SA \\
9 & 8 & 5 & 5 & 1
\end{tabular}

5. We’d like more exploration on macro assume, like assert

\hspace{6mm}
\begin{tabular}{lllll}
SF & F & N & A & SA \\
0 & 0 & 1 & 10 & 16
\end{tabular}

6. We’d like more exploration on keyword such as one of \tcode{unsafe_assume} / \tcode{assume} / \tcode{__assume} / \tcode{_Assume} / …

\hspace{6mm}
\begin{tabular}{lllll}
SF & F & N & A & SA \\
5 & 7 & 9 & 5 & 2
\end{tabular}

7. We’d like more exploration on magic library function such as \tcode{std::assume(expr)}.

\hspace{6mm}
\begin{tabular}{lllll}
SF & F & N & A & SA \\
0 & 0 & 0 & 9 & 14
\end{tabular}

\subsection{SG21, Prague (February 2020)}

Assumptions should proceed independently of contracts.

\hspace{6mm}
\begin{tabular}{lllll}
SF & F & N & A & SA \\
9 & 8 & 5 & 6 & 5
\end{tabular}

\subsection{EWG, online telecon (2021-12-02)}

1. In D1774R5, we should spell the assume as \tcode{[[assume: expr]]}.

\hspace{6mm}
\begin{tabular}{lllll}
SF & F & N & A & SA \\
0 & 0 & 1 & 12 & 5
\end{tabular}
\hspace{5mm}Consensus against

2. In D1774R5, we prefer assume’s parameter to be just an “attribute-grammar-conforming token soup”, not an expression.

\hspace{6mm}
\begin{tabular}{lllll}
SF & F & N & A & SA \\
0 & 0 & 2 & 8 & 6
\end{tabular}
\hspace{5mm}Consensus against

3. Send D1774R5 to electronic polling for forwarding to CWG for inclusion in C++23, in Bucket 2.

\hspace{6mm}
\begin{tabular}{lllll}
SF & F & N & A & SA \\
6 & 8 & 5 & 0 & 0
\end{tabular}
\hspace{5mm}Consensus

\section{Proposed wording}

Add the following subclause to [dcl.attr]:

\begin{adjustwidth}{0.5cm}{0.5cm}
\begin{addedblock}
\subsection*{Assumption attribute \hspace{7.33cm} [dcl.attr.assume]}

The \textit{attribute-token} \tcode{assume} may be applied to a null statement; such a statement is an assumption. An \textit{attribute-argument-clause} shall be present and shall have the form:

\hspace{5mm}\tcode{( }\textit{assignment-expression}\tcode{ )}

The expression shall be contextually convertible to \tcode{bool} [conv.general]. The expression is not evaluated. If the converted expression would not evaluate to \tcode{true}, the behavior is undefined.

\begin{note}
The expression is potentially evaluated [basic.ref.odr]. The use of assumptions is intended to allow implementations to analyze the form of the expression and deduce information used to optimize the program.
\end{note}

\begin{example}
\begin{codeblock}
int divide_by_32(int x)  {
    [[assume(x >= 0)]];
    return x/32;   // The instructions produced for the division
                   //  may omit handling of negative values
}
\end{codeblock}
\end{example}
\end{addedblock}
\end{adjustwidth}

Modify [expr.const] as follows:

\begin{adjustwidth}{0.5cm}{0.5cm}
If $E$ satisfies the constraints of a core constant expression, but evaluation of $E$ would evaluate an operation that has undefined behavior as specified in [library] through [thread] of this document, \added{a statement with an assumption ([dcl.attr.assume]) whose converted \textit{assignment-expression} would not evaluate to \tcode{true}, }or an invocation of the \tcode{va_­start} macro ([cstdarg.syn]), it is unspecified whether \tcode{e} is a core constant expression.

For the purposes of determining
whether an expression $E$ is a core constant expression,
the evaluation of a call to a member function of \tcode{std::allocator<T>}
as defined in [allocator.members], where \tcode{T} is a literal type,
does not disqualify $E$ from being a core constant expression,
even if the actual evaluation of such a call
would otherwise fail the requirements for a core constant expression.
Similarly, the evaluation of a call to
\tcode{std::construct_at} or \tcode{std::ranges::construct_at}
does not disqualify $E$
from being a core constant expression unless
the first argument, of type \tcode{T*}, does not point
to storage allocated with \tcode{std::allocator<T>} or
to an object whose lifetime began within the evaluation of $E$, or
the evaluation of the underlying constructor call
disqualifies $E$ from being a core constant expression.
\added{Further, a statement with an assumption ([dcl.attr.assume]) whose converted \textit{assignment-expression} is itself not a core constant expression does not disqualify $E$ from being a core constant expression.}
\end{adjustwidth}

\section*{Document history}

\begin{itemize}
\item \textbf{R0}, 2019-06-17: Initial version.
\item \textbf{R1}, 2019-10-06: Updated text to reflect removal of Contracts from C++20; made proposed attribute syntax backwards-compatible by replacing colon with parentheses.
\item \textbf{R2}, 2019-11-25: Changed title to ``Portable assumptions''; changed semantics from UB if expression would evaluate to \tcode{false} to UB if expression would \emph{not} evaluate to \tcode{true}; changed syntax section to propose attribute-syntax only, dropping ``magic'' library function syntax as a viable alternative.
\item \textbf {R3}, 2020-01-13: Updated text to clarify the discussion of the proposed semantics and syntax.
\item \textbf{R4}, 2021-11-15:  Added wording. Added polls. Added code size measurement results. Updated and restructured text, adding discussion of proposed semantics and recent related work.
\item \textbf{R5}, 2021-12-02: Updated wording (removed feature-test macro, allowed duplicate attributes, added clarifications). Updated and restructured text, expanding semantics section to reflect discussion in EWG and on the WG21 reflectors. %TODO Added Peter Dimov's smart pointer code example.
\end{itemize}

\section*{Acknowledgements}

Many thanks to Herb Sutter, Chandler Carruth, Joshua Berne, Michael Spencer, Jonathan Caves, Hal Finkel, Erich Keane, Judy Ward, Inbal Levi, Eric Brumer, Nathan Sidwell, Daveed Vandevoorde, Jens Maurer, Ga\v sper A\v zman, Peter Dimov, Gabriel Dos Reis, Arthur O'Dwyer, Aaron Ballman, and Martin Uecker for their help with this proposal.

\renewcommand{\bibname}{References}
\bibliographystyle{abstract}
\bibliography{ref}

\end{document}